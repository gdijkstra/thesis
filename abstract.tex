\chapter*{Abstract}

In type theory defining new types is often done by giving an inductive
definition. Many variants of inductive definitions have been invented
over the years, ranging from ordinary definitions such as the natural
numbers to inductive definitions that are defined mutually with other
inductive definitions as well as functions, as used to describe the
syntax of type theory itself. In the recent years, higher inductive
types have been proposed as a means to inductively define types in
homotopy type theory with non-trivial higher structure. The innovation
of higher inductive types is to also consider constructors that create
equations, or paths, in the inductive type.

In this thesis we present a theory of \emph{quotient
  inductive-inductive definitions}, which are inductive-inductive
definitions extended with path constructors, truncated to sets. The
resulting theory is an improvement over previous treatments of
inductive-inductive and indexed inductive definitions in that it
unifies and generalises these into a single framework.

We give the type of specifications of quotient inductive-inductive
definitions mutually with its interpretation as categories of
algebras. A categorical characterisation of the induction principle is
given and is shown to coincide with the property of being an initial
object in the categories of algebras. From the categorical
characterisation of induction, we derive a more type theoretic
induction principle for our quotient inductive-inductive definitions
that looks like the usual induction principles.

The existence of initial objects in the categories of algebras
associated to quotient inductive-inductive definitions is established
for a class of definitions. This is done by a colimit construction
that can be carried out in type theory itself in the presence of
natural numbers, coproducts and quotients or equivalently,
coequalisers.

Finally, we look at the problems we run into when trying to generalise
our theory to one of higher inductive types.



