\chapter*{Abstract}

In this thesis we present a theory of \emph{quotient
  inductive-inductive definitions}, which are inductive-inductive
definitions extended with constructors for equations. The resulting
theory is an improvement over previous treatments of
inductive-inductive and indexed inductive definitions in that it
unifies and generalises these into a single framework. The framework
can also be seen as a first approximation towards a theory of higher
inductive types, but done in a set truncated setting.

We give the type of specifications of quotient inductive-inductive
definitions mutually with its interpretation as categories of
algebras. A categorical characterisation of the induction principle is
given and is shown to coincide with the property of being an initial
object in the categories of algebras. From the categorical
characterisation of induction, we derive a more type theoretic
induction principle for our quotient inductive-inductive definitions
that looks like the usual induction principles.

The existence of initial objects in the categories of algebras
associated to quotient inductive-inductive definitions is established
for a class of definitions. This is done by a colimit construction
that can be carried out in type theory itself in the presence of
natural numbers, sum types and quotients or equivalently,
coequalisers.
