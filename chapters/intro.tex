\chapter{Introduction}
\label{intro}

In this thesis we set out to develop a theory of \emph{quotient
  inductive-inductive definitions}, which are inductive-inductive
definitions \cite{NordvallForsberg2013} extended with path
constructors. This first chapter we will give some context of the
problem and discuss prior art and related concepts. The chapter is
concluded by an overview of the thesis and a list of contributions.

\section{Induction in mathematics}
In mathematics, induction is an important proof technique. The most
common and perhaps oldest form of induction is induction on the
natural numbers, dating back to at least Plato
\cite{Acerbi2000}. Induction on the natural numbers gives us a way
want to prove that a formula $\phi(n)$ holds for any $n \in \natty$:
we have to prove $\phi(0)$ and prove that for any $n \in \natty$ that
$\phi(n)$ implies $\phi(n+1)$. In fact, the fact that the natural
numbers satisfy this property can be seen as one of the defining
properties of the natural numbers. This was first written down
formally by Guiseppe Peano \cite{Peano1889}. He defined the natural
numbers to be a set $\natty$ with the properties such as:
\begin{itemize}
\item $0 \in \natty$
\item for any $n \in \natty$, $\natsucc(n) \in \natty$.
\item $\natty$ satisfies the induction principle
\end{itemize}
The remaining axioms describe the equality relation on the natural
numbers and postulate the injectivity of the $\natsucc$ function
symbol and that $\natzero \neq \natsucc\ n$ for all $n \in \natty$.

A consequence of the induction principle for natural numbers, is that
we can define functions $\natty \to X$, for some set $X$,
\emph{recursively}: it suffices to define $f\ 0$ and
$f\ (\natsucc(n))$, where we may refer to $f\ n$. We can use this to
define addition $n + m$ on the natural numbers, by recursion on the
second argument $m$, \ie we define $n + 0 \ddefeq n$ and
$n + \natsucc(m) \ddefeq \natsucc(n+m)$.

In mathematics, the construction of new sets is often done by taking
the natural numbers as a given and building upon this and quotienting
where needed, as opposed to giving an inductive definition
directly. For example, the rational numbers can be constructed as
$\natty \times \natty$ quotiented by the relation
$(a , b) \sim (c , d)$ if and only $ad = bc$. When quotienting
infinite sets is not always unproblematic. The usual construction of
the real numbers as a quotient of Cauchy sequences of rational numbers
requires the axiom of choice to show that it forms a complete metric
space. Direct inductive definitions (with equations) can avoid such
problems, as we will see in \cref{cauchy-reals}.

\section{Induction in computer science}

Recursion is a central concept to computer science. Data structures
are often defined in terms of themselves: for example, a binary tree
is either a leaf or a pair of binary trees. Functional programming
languages therefore often come with a mechanism to express definitions
such as these, usually under the name of \emph{algebraic data types}.

In Haskell \cite{Jones2003}, one can define (linked) lists as the
algebraic data type:
$$
\data\ \List\ a = \listnil\ |\ \listcons\ a\ (\List\ a)
$$
As opposed to having a recursion principle associated with the
algebraic data type, we have \emph{pattern matching} and general
recursion. For example, we can define a function as follows:
%
\begin{align*}
  &\listmap : (a \to b) \to \List\ a \to \List\ b \\
  &\listmap\ f\ \listnil = \listnil \\
  &\listmap\ f\ (\listcons\ x\ xs) = \listcons\ (f\ x)\ (\listmap\ f\ xs)
\end{align*}
%
Pattern matching and general recursion are powerful enough to let us
implement the recursion operator associated with the inductive
type. In the case of lists, this is usually called $\listfoldr$:
%
\begin{align*}
  &\listfoldr : b \to (a \to b \to b) \to \List\ a \to b \\
  &\listfoldr\ e\ op\ \listnil = e \\
  &\listfoldr\ e\ op\ (\listcons\ x\ xs) = op\ x\ (\listfoldr\ e\ op\ xs)
\end{align*}
%

If we care about the totality of our definitions, pattern matching and
general recursion are too powerful. First of all we have to restrict
ourselves to \emph{structurally} recursive definitions: recursive
calls may only be done on subterms of the patterns on the left hand
side of the pattern matching clause. This is however not enough to
ensure termination of definitions. The inductive types themselves also
have to be of the right shape:: they have to be \emph{strictly
positive}. If we were to have an inductive type:
%
\begin{datatype}{\Tty}{\Type}
  \constr{\Ta}{(\Tty \to \Tty) \to \Tty}
\end{datatype}
then we can define:
\begin{align*}
  &uh : T \to \emptyty \\
  &uh\ (\Ta\ f) \ddefeq uh\ (f\ (\lambda x . x)) \\
  \\
  &oh : T \\
  &oh \ddefeq \Ta\ (\lambda x . x)
\end{align*}
%
Since $f\ (\lambda x . x)$ is structurally smaller than $\Ta\ f$, the
definition of $uh$ is structurally recursive. However, it does give us
a term that does not have a normal form, namely $uh\ oh$.

Algebraic data types allow us to specify types or a family of types
parametrised by type variables. As these are \emph{parameters}, we are
not allowed to vary them in the result type of the
constructors. Lifting this restriction, \ie turning the parameters
into \emph{indices}, gives us \emph{generalised algebraic data types}
(GADTs) or \emph{inductive families}. We can use the indices to store
extra information in the type, allowing us to encode invariants. They
have been used to implement a type of well-typed abstract syntax trees
\cite{Pavsalic2004} and red-black trees \cite{Kahrs2001}, among many
other uses.

Inductive families are especially powerful in a dependently typed
setting in conjunction with \emph{dependent pattern matching}
\cite{Coquand1992}, such as it is implemented in
Agda\footnote{\url{http://wiki.portal.chalmers.se/agda/pmwiki.php}}
\cite{Norell2007}. The information encoded in the indices may tell us
that certain cases are impossible and need not be treated, or they may
tell us that certain variables in the patterns are equal. We can
define the type of length-indexed lists, also referred to as vectors
as follows:
%
\begin{datatype}{\vecty\ (A : \Type)}{\natty \to \Type}
  \constr{\vecnil}{\vecty\ A\ \natzero} \\
  \constr{\veccons}{A \to (n : \natty)\ (xs : \vecty\ A\ n) \to \vecty\ A\ (\natsucc\ n)}
\end{datatype}
%
We can define a function that returns the first element of a non-empty
list as follows:
%
\begin{align*}
  &\vechead : (A : \Type)\ (n : \natty) \to \vecty\ A\ (\natsucc\ n) \to A \\
  &\vechead\ A\ n\ (\veccons\ x\ .n\ xs) = x
\end{align*}
%
By pattern matching on the argument of type
$\vecty\ A\ (\natsucc\ n)$, we get two cases: the value is either
constructed with constructor $\vecnil$ or $\veccons$. If we unify the
type of the argument with that of the constructors, we see that the
argument can never be $\vecnil$, as that produces something of type
$\vecty\ A\ \natzero$. Furthermore, if we consider the $\veccons$
case, we notice that the natural number argument of the constructor
has to coincide with the one we already had in our context, hence we
get the non-linear pattern $\vechead\ A\ n\ (\veccons\ x\ .n\ xs)$,
where the dot indicates that it is a repeated variable.

It has been shown that under certain assumptions, \ie if the data
types are strictly positive and the recursion is structural, along
with certain assumptions about the equality in the type theory, then
these dependent pattern matching definitions can be translated to ones
using only elimination principles \cite{Goguen2006}. The restrictions
on the type of equality used can be relaxed in certain cases
\cite{Cockx2014}, giving us a form of dependent pattern matching which
is compatible with homotopy type theory.

In Agda, apart from inductive families, we can also define inductive
types mutually with other inductive types/families or functions,
giving us inductive-inductive \cite{NordvallForsberg2013} and
inductive-recursive definitions \cite{Dybjer1999} respectively.

\section{Formal treatment of induction}

So far we have given several examples of inductive sets and types and
recursive definitions, but have not given a formal definition of what
an inductive definition is. In \cite{Aczel1977}, the author writes:

\begin{quote}
  ``Inductive definitions of sets are often informally presented by
  giving some rules for generating elements of the set and then adding
  that an object is to be in the set only if it has been generated
  according to the rules.''
\end{quote}

In \cite{MartinLof1971}, the author gives a scheme of the kind of
rules that comprise inductive definitions in first-order logic. In
this thesis we are concerned with inductive definitions in
Martin-L\"of Type Theory \cite{MartinLof1972}, but we will also look
at category theoretic characterisations in \cref{in-category-theory},
to guide us to appropriate generalisations.

\subsection{In type theory}
\label{in-type-theory}
Extending a type theory with a particular inductive definition means
that we have the extend the theory with four sets of inference rules:
%
\begin{itemize}
\item \emph{type formation rules} ($\natty$ is a type)
\item \emph{introduction rules} ($\natzero : \natty$, $\natsucc : \natty \to \natty$)
\item \emph{elimination rules} (given $P : \natty \to \Type$, $m_{\natzero} : P\ \natzero$, $m_{\natsucc} : (n : \natty) \to P\ n \to P\ (\natsucc\ n)$, we get $\natind\ P\ m_{\natzero}\ m_{\natsucc} : (x : \natty) \to P\ x$)
\item \emph{computation rules} ($\natind\ P\ m_{\natzero}\ m_{\natsucc}\ \natzero = m_{\natzero}$, $\natind\ P\ m_{\natzero}\ m_{\natsucc}\ (\natsucc\ n) = m_{\natsucc}\ n\ (\natind\ P\ m_{\natzero}\ m_{\natsucc} n)$)
\end{itemize}
%
In the case of the natural numbers, the type formation, introduction
and elimination rules are not essentially different from Peano's
rules. Missing however are rules defining an equality relation on
$\natty$. As we will see, we can define the equality relation as a
single parametric inductive definition uniformly for all types. Given
this notion of equality and given that we have a universe of types
available, one can derive that the constructors are disjoint and
injective.

The declaration of an inductive definition involves given rules in all
these classes. However, as observed by Backhouse et
al. \cite{Backhouse1989}, it is enough to give the type formation
rules and introduction rules: the elimination principle along with its
computation rules can be derived from them. This fact is also
reflected in how one declares inductive definitions in type
theory-based proof assistants such as Coq \cite{Bertot2004} and Agda
by simply giving a sequence of constructors, \ie introduction rules.

\subsection{In category theory}
\label{in-category-theory}

Inductive definitions can be characterised in category theory as
\emph{initial $F$-algebras}, for some endofunctor $F$ on an
appropriately chosen category. For example, the set of natural numbers
with its operations $\natzero$ and $\natsucc$ form an initial algebra
for the functor $F X \ddefeq \unitty + X$. The property of being an
initial algebra contains the same information as given by the four
classes of rules, for the appropriately chosen endofunctor. 

The perspective on inductive definitions as initial algebras allows us
to generalise easily. If we associate ordinary inductive types with
initial algebras of endofunctors on a category $\Cc$, which is a model
of type theory, \eg $\Set$, then inductive families correspond to
initial algebras of endofunctors on slice categories of
$\Cc$. 


Another way to generalise is based on the observation that
$F$-algebras for an endofunctor $F$ coincide with $F^*$-monad
algebras, where $F^*$ is the free monad of $F$. (Note that $F^*$ may
not exist: $F$ needs to be a strictly positive functor.) As is
described in \cite{Shulman2011}, we can interpret this as ordinary
inductive types being associated with free monads. Generalising these
inductive types would be the same as considering a larger class of monads. 

As monads and monad algebras are also used to talk about algebraic
theories, such as the theory of groups, the aforementioned observation
makes clear the relationship between inductive definitions and
algebraic theories. An essential ingredient of algebraic theories is
the ability to talk about equations. This is something which is
lacking in the inductive definitions we have seen so far.

\section{Higher inductive types and homotopy type theory}

Higher inductive types are a generalisation of inductive types,
stemming from homotopy type theory, where apart from the usual
constructors, called \emph{point constructors}, we may also have
equations as constructors, called \emph{path constructors}.

Before we go on and give some examples, we will give a brief recap of
homotopy type theory. For a more in depth introduction, we refer the
reader to the usual book \cite{UFP2013}.

In type theory, we can define an equality relation on any type
inductively, as follows: suppose $A$ is a type, we define:
%
\begin{datatype}{\_=_A\_}{A \to A \to \Type}
  \constr{\refl}{(x : A) \to x =_A x}
\end{datatype}
%
If we have a term $\refl\ x : x =_A y$ for some $x, y : A$ then this
only type checks if $x$ is \emph{definitionally} equal to $y$, \ie
they are the same up to $\beta$- and $\eta$-equality (and possibly
more equalities). 

The equality defined above is usually referred to as
\emph{propositional} equality. This name comes from the idea that
equality is propositional, \ie any two terms of that type are
equal. If we normalise a closed term of an identity type $x = y$, it
normalises to $\refl$, hence it gives us that $x \equiv y$. As such,
one would expect that if we have two terms $p, q : x = y$, then also
$p = q$, \ie we have \emph{uniqueness of identity proofs}. Using
dependent pattern matching this is easy to prove:
%
\begin{align*}
  &\uip : (A : \Type)\ (x\ y : A)\ (p\ q : x = y) \to p = q \\
  &\uip\ A\ x\ .x\ (\refl\ x)\ (\refl\ x) \ddefeq \refl\ (\refl\ x)
\end{align*}
%
However, if we consider the induction principle for this type, called
the $J$ rule, this is not so clear at all. What we can show with $J$
is that given a type $A$, the relation $=_A$ is an equivalence
relation. Furthermore we can show that it forms a groupoid with
transitivity as its binary operation, reflexivity as its unit and
symmetry as its inverse operation. Using $J$, we can show that these
operations all satisfy the groupoid laws up to propositional equality
again, but not definitionally. Using this idea of types as groupoids,
a model of type theory has been given in the category of groupoids
\cite{Hofmann1998}. Since there are non-trivial groupoids, the
groupoid model contains types which refute the uniqueness of identity
types principle.

The story does not end with groupoids, however. Since the groupoid
laws are satisfied up to propositional equality, a type itself again,
we get a tower of groupoids: we get \inftygrpds. \inftygrpds are also
the object of study in homotopy theory: they can be thought of as
topological spaces up to homotopy. Types can therefore be thought of
as \inftygrpds \cite{VanDenBerg2011,Lumsdaine2009} and therefore also
as topological spaces up to homotopy. This correspondence leads to a
geometric intuition for type theory: types can be seen as spaces with
its identity types as the paths in said type.

One important axiom that is considered in homotopy type theory, is the
univalence axiom, which roughly tells us that isomorphic types are
also propositionally equal. This axiom is inspired by and holds in the
simplicial set model of type theory \cite{Kapulkin2012}. A univalent
universe is then one example of a type that does not satisfy
uniqueness of identity proofs: we may have different isomorphisms,
giving rise to different propositional equalities between types. An
example of this are the identity map and the negation map on the
booleans: these are two distinct isomorphisms, yielding by univalence
two distinct propositional equalities $\boolty = \boolty$. In fact, if
we have a hierarchy of univalent universes
$\Type_0 : \Type_1 : \Type_2 : \hdots$, then $\Type_{n+1}$ has a
strictly more complicated higher equality structure
\cite{Kraus2015ii}.

Apart from univalent universes of types invalidating uniqueness of
identity proofs, there are also plenty of geometric examples to take
from homotopy theory, such as the circle and the torus. As we have
seen, we define new data types in type theory usually as an inductive
type. However, ordinary inductive definitions do not give us a way to
create new paths between points that were not already there. The
solution is to generalise the idea of constructor to also allow for
paths to be constructed. We can then define the circle, which is just
a point with a non-trivial loop, as the following \emph{higher
  inductive type}:
\begin{datatype}{\circlety}{\Type}
  \constr{\circlebase}{\circlety} \\
  \constr{\circleloop}{\circlebase = \circlebase}
\end{datatype}
In higher inductive types, we have the ordinary constructors, such as
$\circlebase$ in this example. These are called \emph{point
  constructors}, as they can be thought of as constructing points in
the space we are defining inductively. The constructor $\circleloop$
is a \emph{path constructor}: it constructs a new path from
$\circlebase$ to $\circlebase$.

The induction principle for the circle is as follows:
$$
\circleind : (P : \circlety \to \Type)\ (m_{\circlebase} : P\ \circlebase)\ (m_{\circleloop} : \pathover{P}{\circleloop}{m_{\circlebase}}{m_{\circlebase}}) \to (x : \circlety) \to P\ x
$$
To show that $P$ holds for any $x : \circlety$, we need to show that
it holds for the base point and that transporting this value along the
loop is equal to the value itself. 

The induction principle for the circle can also be used to show that
the $\circleloop$ is in fact a non-trivial loop, \ie
$\circleloop \neq \refl_{\circlebase}$. To prove this, we need to have
a universe at hand which is univalent. This is similar to how we
cannot show that $\boolt \neq \boolf$, if there is no universe to
eliminate into, with the additional requirement of that universe being
univalent.

Apart from allowing constructors to construct paths between points,
higher inductive types also allow for \emph{higher} path constructors
which construct paths between paths. For example, we can describe the
torus as the following higher inductive type:
%
\begin{datatype}{\torusty}{\Type}
  \constr{\torusbase}{\torusty} \\
  \constr{\torusp}{\torusbase = \torusbase} \\
  \constr{\torusq}{\torusbase = \torusbase} \\
  \constr{\torusr}{\torusp \ct \torusq = \torusq \ct \torusp}
\end{datatype}
%
The approach of using higher inductive types to define these
topological spaces (up to homotopy) has been used successfully to
formalise various results of homotopy theory in type theory, leading
to the field of \emph{synthetic homotopy theory}. Formalisations
include the fundamental group of circle \cite{Licata2013} and more
general results on the homotopy groups of spheres
\cite{Licata2013ii,Brunerie2016}, the Blakers-Massey theorem
\cite{Favonia2016} and the Mayer-Vietoris sequence \cite{Cavallo2015}.

Applications aside, as of yet a general definition and theory of
higher inductive types is still lacking. There is a note on the
semantics of higher inductive types \cite{Lumsdaine2013}, \ie a
metatheoretic and semantic treatment. The ideas in this thesis are
very much inspired by this note. In \cite{Sojakova2014}, the author
gives an internal specification of a class of higher inductive types
called W-suspensions, \ie the author defines an induction principle
and the category of algebras and shows that initiality in the category
of algebras coincides with satisfying the induction principle. This is
all carried out inside type theory itself, which is also a goal of
this thesis.

\section{A theory of quotient inductive-inductive definitions}

Our goal in this thesis is to give a theory of quotient
inductive-inductive definitions, which can be seen as a stepping stone
towards a theory of higher inductive types. Quotient
inductive-inductive types can either be seen as a generalisation of
inductive-inductive types or as a subclass of higher
inductive-inductive types. We extend inductive-inductive definitions
with path constructors, but we truncate the result to be a set: our
definitions satisfy uniqueness of identity proofs. As such, they can
be seen as a subclass of higher inductive-inductive types. Since we
work with sets, we only consider path constructors that construct
paths between points, as any higher path constructors would not add
anything new.

Another goal of this thesis is to give a theory that can be expressed
inside type theory itself. While formalising the theory is already a
good idea on its own, having an internal definition of the inductive
definitions means that anything we prove about them internally
automatically holds for all models of type theory. Furthermore, if the
theory happens to be expressible in a small core theory and we manage
to construct the inductive types also in this core, we have an
implementation of our inductive definitions in this small type theory.

From a programming point of view, having the specification of
inductive definitions as a first-class citizen in your language allows
for meta-programming: we can write generic programs by recursion on
the structure of the inductive definitions \cite{Benke2003}.

Working on the theory of inductive definitions internal to type theory
is not without problems, however. When we talk about the computation
rules of type theoretic inductive definitions, we talk about
\emph{definitional} equalities. These definitional equalities have
none of the non-trivial structure that its internal counterpart,
propositional equality, may have. They are equalities \emph{on the
  nose}. In type theory itself, we can only talk about the weaker
notion of propositional equality. This complicates the situation
already when we talk about ordinary inductive types: in
\cite{Awodey2012} the authors internally prove for W-types that
initiality and the induction principle coincide. Doing so requires
some involved reasoning about equalities between propositional
equalities: there are so called coherence problems one needs to
solve. In this thesis we observe in \cref{untruncated} that these
coherence problems grow in number with the number of constructors,
even if they are all just point constructors.

Dealing with these coherences in a uniform way means that we have to
talk about \inftycats \cite{Camarena2013}. Formalising the usual
notions of \inftycat in type theory requires us to work with
simplicial sets in type theory. However, we want to work with
\emph{types}, not simplicial sets. A more promising approach is
extending type theory with a strict equality
\cite{Altenkirch2016ii,Altenkirch2016iii}. In this thesis, we will
deal with the coherences by simply working with the truncated types:
we work with sets instead.

\section{Related work}

\todoi{Fix!}

\section{Overview of the thesis and contributions}

The thesis is organised as follows:

\begin{itemize}
\item In \cref{prelims}, we give the basic concepts and notation we
  will use in this thesis, along with proofs of basic propositions and
  lemmata used throughout the thesis.
\item \Cref{qits} presents examples of quotient inductive-inductive
  definitions and their applications and compares them to other
  notions such as quotients and coequalisers.
\item \Cref{describing} contains the formal specification of quotient
  inductive-inductive definitions. They are specified as being a
  certain kind of iterated dialgebras, which generalises the
  presentation of ordinary inductive definitions as algebras of
  functors.
\item Having a formal specification of quotient inductive-inductive
  definitions and the corresponding categories of algebras, we show in
  \cref{induction} that the initial algebra semantics coincides with a
  categorical formulation of the type theoretic induction
  principle. From the categorical formulation we derive the type
  theoretic formulation, showing that induction and initiality
  coincide for our inductive definitions.
\item In \cref{constructing} we give some preliminary results on
  constructing quotient inductive-inductive definitions given some
  reasonable assumptions on the type theory.
\item \Cref{untruncated} is about what the difficulties are when
  moving from a set-based setting to the untruncated case. Here we go
  into detail about the coherence issues we face and how some of these
  can be solved in an ad-hoc manner.
\item The last chapter, \cref{conclusion}, presents the conclusions of
  the thesis along with a discussion of the presented results. We also
  point out several avenues for future work.
\end{itemize}

\subsection{List of main contributions}

\todoi{Fix!}

The main contributions of this thesis are:

\begin{itemize}
\item a formal specification of quotient inductive-inductive
  definitions, alongside with their categorical interpretation, given
  in type theory: def. blablba
\item a proof of the logical equivalence of initiality and the
  category theoretic induction principle: thm blabla
\item a derivation of the type theoretic induction principle from the
  category theoretic one
\item construction of initial algebras for a class of quotient inductive types
\end{itemize}

\section{Declaration of authorship}

\todoi{Declare authorship}
