\chapter{Containers for quotient inductive-inductive definitions}

The inductive definitions therefore need to be \emph{strictly}
positive: positivity alone does not suffice. There are different ways
to formally specify strictly positive functors. We can use a syntactic
way to describe them as the class of functors that contains all
constant functor, closed under sums and products of strictly positive
functors, exponentiation with a constant on the left of the arrow, and
taking fixpoints \cite{Morris2007}. A more compact way to characterise
strictly positive functors on $\SET$ in type theory is as
\emph{containers}:
%
\begin{definition}
\label{def-container}
  A \emph{container} on $\SET$ consists of:
  \begin{itemize}
  \item $S : \Set$, a type of \emph{shapes}
  \item $P : S \to \Set$, a family of \emph{position}, indexed by the shapes.
  \end{itemize}
  %
  The container with shapes $S$ and positions $P$ is denoted as
  $\cont{S}{P}$
\end{definition}
%
The corresponding functor is called the \emph{extension} of the container:
%
\begin{definition}
  Given a container $\cont{S}{P}$, its \emph{extension} is the functor
  $\context{\cont{S}{P}} : \Func{\SET}{\SET}$ with its action on objects
  defined as, for every $X : \Set$:
  $$
  \context{\cont{S}{P}}\ X \ddefeq (s : S) \times (P\ s \to X)
  $$
  and its action on functions $f : X \to Y$:
  $$
  \context{\cont{S}{P}}\ f \ddefeq \lambda (s , t) . (s , f \circ t)
  $$
\end{definition}

\section{Containers for $\SET$-sorted definitions}
To give the data for a quotient inductive-inductive definition, we
often need more than just endofunctors on $\SET$. We are generally
working with functors $\Func{\Alg_s}{S_i}$ where $s : \specty$
describes the previous constructors and $S_i$ is sort category
describing the sort of the constructor we are defining. Containers
have a generalisation to \emph{indexed containers} which describe
functors between slice categories of $\SET$. This concept is again an
instance of the more general notion of \emph{polynomial functor},
which describes strictly positive functors between slices categories
of a locally cartesian closed category. We cannot expect $\Alg_s$ to
be locally cartesian closed in general: if we take $s$ to be the
specification corresponding to setoids (as in
\cref{application-to-equational-theories}), then $\Alg_s$ is equivalent
to the category setoids, which is not locally cartesian closed
\cite{Altenkirch2012}.

If we look at containers a bit more closely, we see that they are
coproduct of a family of representable functors. This observation
leads us to \emph{generalised containers}, also known as
\emph{familially representable functors} \cite{Carboni1995}:
%
\begin{definition}
\label{def-gen-container}
  A \emph{generalised container} on a category $\Cc$ consists of:
  \begin{itemize}
  \item $S : \Set$, a type of shapes,
  \item $P : S \to | \Cc |$, a family of representing objects, indexed
    by the shapes.
  \end{itemize}
\end{definition}
%
The extension generalises straightforwardly:
%
\begin{definition}
  Given a container $\cont{S}{P}$ on a category $\Cc$, its extension
  is the functor $\context{\cont{S}{P}} : \Func{\Cc}{\SET}$ with its
  action on objects defined as, for every $X : | \Cc |$:
  $$
  \context{\cont{S}{P}}\ X \ddefeq (s : S) \times \Cc(P\ s, X)
  $$
  and its action on functions $f : X \to Y$:
  $$
  \context{\cont{S}{P}}\ f \ddefeq \lambda (s , t) . (s , f \circ t)
  $$
\end{definition}
%
For ordinary containers, there exists a composition operation on
them. For generalised containers, such an operation only makes sense
if the extensions are composable. Even then, the usual construction is
not always possible: suppose we have a $\Cc$-container $\cont{S}{P}$
and a $\SET$-container $\cont{T}{Q}$, given an object $X : | \Cc |$, we can calculate:
\begin{align*}
  \context{\cont{T}{Q}}\ (\context{\cont{S}{P}}\ X) &=& &(t : T) \times (Q\ t \to \context{\cont{S}{P}}\ X) \\
  &=& &(t : T) \times (Q\ t \to (s : S) \times \Cc(P\ s, X)) \\
  &=& &(t : T) \times (a : Q\ t \to S) \times (b : (x : Q\ t) \to \Cc(P\ (a\ x), X))
\end{align*}
Looking at the above computation, a candidate $\Cc$-container whose
extension is equal to the composite
$\context{\cont{T}{Q}} \circ \context{\cont{S}{P}}$ seems to have as shapes
$$
(t : T) \times (Q\ t \to S)
$$
In the ordinary container case, to derive the corresponding positions,
one uses currying on the type of $b$. However, if
$\context{\cont{S}{P}}$ happens to have a left adjoint
$L : \Func{\SET}{\Cc}$, then the situation simplifies as follows:
\begin{align*}
  \context{\cont{T}{Q}}\ (\context{\cont{S}{P}}\ X) &=& &(t : T) \times (Q\ t \to \context{\cont{S}{P}}\ X) \\
  &=& &(t : T) \times (\Cc(L (Q\ t), X)) \\
\end{align*}
So we get the following equation:
$$
\context{\cont{T}{Q}} \circ \context{\cont{S}{P}} = \context{\cont{T}{L \circ P}}
$$

The notion of \emph{container morphism}, which are used to represent
natural transformations between containers can be phrased as follows
for generalised containers:
\begin{definition}
  Given $\Cc$-containers $\cont{S}{P}$ and $\cont{T}{Q}$, a container
  morphism consists of:
  \begin{itemize}
  \item $f : S \to T$
  \item $g : (s : S) \to \Cc(Q\ (f\ s), P\ s)$
  \end{itemize}
  with its extension being the natural transformation:
  \begin{align*}
    &\context{f , g} : (X : | \Cc |) \to \context{\cont{S}{P}}\ X \to \context{\cont{T}{Q}}\ X \\
    &\context{f , g}\ X\ (s , t) \ddefeq (f\ s , t \circ (g\ s))
  \end{align*}
  Naturality follows from the associativity law of $\Cc$.
\end{definition}

\section{Containers for arbitrarily sorted definitions}
We have given a way to describe strictly positive functors and natural
transformations needed to describe $\SET$-sorted quotient
inductive-inductive definitions. However, the functors we work with
are not generally functors into $\SET$, but may also be into any sort
category.

In this section we will show how this can be done for the special case
$\FAM$. Suppose we have a category $\Cc$, which we can think of as
being a category of $\FAM$-sorted algebras. It is therefore equipped
with a forgetful functor $U : \Func{\Cc}{\FAM}$. Describing the
arguments of a $\FAM$-sorted constructor over $\Cc$ requires us to
give a functor $F : \Func{\Cc}{\FAM}$ such that
$t_1 \circ F = t_1 \circ U$, where $t_1 : \Func{\FAM}{\SET}$ is its
forgetful functor.

Note that we have $\FAM = \SET^I$, therefore by the
cartesian-closedness of $\Cat$, we have
$\Func{\Cc}{\SET^I} = \Func{\Cc \times I}{\SET}$. To give a functor
$F : \Func{\Cc}{\FAM}$ is to give two functors
$F^0, F^1 : \Func{\Cc}{\SET}$ along with a natural transformation
$\alpha : \Nat{F^1}{F^0}$.

Furthermore, we have the requirement that $F^0 = t_1 \circ U$. If we
assume that we have an adjunction $L \dashv U : \Func{\Cc}{\FAM}$,
$t_1 \circ U$ will also have a left adjoint (as $t_1$ also has a left
adjoint). If $t_1 \circ U$ has a left adjoint, it is a representible
functor, which means it is also a container.

\section{Generalising W-types}

Let $\Cont_{\SET}$ be the type of containers on $\SET$. W-types give
us a primitive $\wty : \Cont_{\SET} \to \Set$ along with a constructor
$\wsup : (F : \Cont_{\SET}) \to \context{F}\ \wty_F \to \wty$. For
every container $F$, $(\wty_F,\wsup_F)$ satisfy the induction
principle, \ie we have:
$$
\welim : (F : \Cont_{\SET}) \to (P : \Fam_{\algcat{F}}\ (\wty_F,\wsup_F)) \to \DepHom_{\algcat{F}}\ P
$$
Generalising this idea to quotient inductive-inductive definitions would be to have a primitive:
$$
\wty : (\Ss : \sortsty)\ (s : \Cont_{\Ss}) \to \Alg_s
$$
with induction principle:
$$
\welim : (\Ss : \sortsty)\ (s : \Cont_{\Ss}) \to (P : \Fam_{\Alg_s}\ (\wty\ \Ss\ s,\wsup\ \Ss\ s)) \to \DepHom_{\Alg_s}\ P
$$
where $\Cont_{\Ss}$ is the same as $\specty_{\Ss}$ but instead of
having arbitrary functors, we have containers instead. The
interpretation of $\Cont_{\Ss}$ as a category of algebras is then also
the same as for $\specty_{\Ss}$, but then composed with the extension
operation on containers.

However, unlike ordinary W-types, this is a rather heavy handed
definition, as $\Cont_{\Ss}$ can be arbitrarily complicated. As such
it would be a bit silly to call such a notion \emph{primitive}. In a
following section we will instead try to construct initial algebras
from more primitive notions, namely the notions of natural numbers and
quotients.

\section{Limitations of containers}

While in the traditional setting, containers (on $\SET$) seem to be an
adequate way to characterise strictly positive functors, it has its
limitations. Let us consider the propositional truncation operation on
$\Set$: $\proptrunc{\_} : \Set \to \Set$. Let $\cont{S}{P}$ be its
container representation, then the following holds:
$$
\unitty = \proptrunc{\unitty} = (s : S) \times (P\ s \to \unitty) = S
$$
Therefore we know that that the shapes $S = \unitty$, hence
$\proptrunc{\_}$ has to be a representable functor. Let $P : \Set$ be
its representing object. $P$ has to either be empty or inhabited. If
it is empty, then we have
$\emptyty = \proptrunc{\emptyty} = \emptyty \to \emptyty = \unitty$, a
contradiction. If it is inhabited, we have
$\unitty = \proptrunc{\boolty} = P \to \boolty$, however
$P \to \boolty$ has at least two distinct inhabitants:
$\lambda x . \boolt$ and $\lambda x . boolf$, also a contradiction.

Now this limitation is not necessarily bad for the expressiveness of
our system. If we wanted to express a constructor of a type $\Aty$
such as $\Ac : \proptrunc{\Aty} \to \Aty$, we could simply ``inline''
propositional truncation, \ie add another sort $\Bty : \Set$ which has
a constructor $\Bd : \Aty \to \Bty$ and a constructor of type
$(x\ y : \Bty) \to x = y$.

