\chapter{Concluding remarks}
\label{conclusion}

In this thesis we have given a formal specification of quotient
inductive-inductive definitions, intended as a stepping stone towards
a theory of higher inductive(-inductive) types. 

\Cref{qits} was devoted to giving examples and intuition for quotient
inductive-inductive types, before diving into the formal
definition. We argued that even only considering higher
inductive(-inductive) types truncated to sets, and therefore naturally
limiting ourselves to point and path constructors, is already a useful
extension over ordinary inductive types. In these examples we
uncovered that quotient inductive-inductive types set itself apart
from ordinary inductive definitions in the following regards:
\begin{itemize}
\item apart from defining a single type, we have a collection of \emph{dependent sorts}
\item any constructor may \emph{refer to any previous constructor}
\item the result type of a constructor may be an \emph{equation} in
  any of the sorts, \ie we allow for \emph{path constructors}
\end{itemize}
Both inductive-inductive definitions as well as higher inductive types
have the property of allowing for references to previous constructors,
which prompted our investigations into a uniform treatment of both
classes of inductive definitions.

In \cref{describing} we give the formal specification of quotient
inductive-inductive definitions, which is given simultaneously with
its interpretation as categories of algebras: we characterise quotient
inductive-inductive definitions roughly as iterated
dialgebras. Dealing with dependent sorts means that ordinary
dialgebras do not suffice: the categories of dialgebras are fibred
over all sorts below the sort of the current constructor. Dealing with
this means we define the category of algebras as an equaliser category
of a category of dialgebras.

The algebras corresponding to path constructors are also given as
objects of an equaliser category, which at least in the set truncated
setting is a natural way of describing them. They are treated
separately from point constructors, however. Future work would be to
find a way to uniformly deal with both point and path constructors,
possibly finding a generalisation of dialgebras suitable for this. We
have attempted such an approach by considering \emph{dependent
  dialgebras}, \ie given $F : \Func{\Cc}{\Set}$ and
$G : \Func{\int_{\Cc} F}{\Set}$ a dependent dialgebra is an object
$X : | \Cc |$ along with:
$$
\theta : (x : FX) \to G (X , x)
$$
This turned out not to be a nice notion to work with, as the category
$\int_{\Cc} F$ is not a well-behaved category to work with for our
purposes. In particular, when defining what morphisms between these
algebras should be, we need to lift a morphism in $\Cc$ to one in
$\int_{\Cc} F$. There is a canonical way of doing so, given an
$x : FX$ for some $X : | \Cc |$, but this operation is not functorial
in the untruncated setting, which makes the approach unattractive if
we want to generalise to higher inductive types.

Given that our definition of quotient inductive-inductive types can be
formulated inside type theory, one avenue for future work would be
applying this definition to generic programming ideas. Having this
definitions as the basis of the implementation of inductive
definitions in your theory is useful when one wants to use
metaprogramming techniques to define programs abstracting over data
types. One aspect of our approach is that we stay with the idea of an
inductive definition being given as a list of constructors, as opposed
to simplifying the situation to being a code of a single
endofunctor. Staying with the list of constructors idea also means
that we could build a system for writing attribute grammars internally
without needing any external tools, allowing for aspect oriented
programming.

Since we have not focussed on our specifications of inductive
definitions being sound in the sense of the categories of algebras
having initial objects, the system can also be used to work with
equational theories in type theory. While there are plenty of
equational theories that are nicely behaved in the sense that the
corresponding category of models has an initial object, this is not
always the case: fields being a notable example.

As we have given a specification of quotient inductive-inductive
definitions, we ought to prove properties about them. In this thesis
we investigate several properties: the logical equivalence of
initiality and induction and the construction of some initial algebras
via sequential colimits. Another important property that should hold
is that the definitions should be invariant under equivalence of
constructors. If we have two specifications $s, s' : \specty$ with the
same dependent sorts, such that if $| \Alg_s | = | \Alg_{s'} |$, \ie
all the constructors combined of $s$ are equivalent to those of $s'$,
then the initial object of $\Alg_s$ should have an isomorphic carrier
to that of $\Alg_{s'}$. This is an important property that is used
often to reason about equivalence of inductive definitions. For
example, it implies that the definitions are invariant under
reordering of constructors. In the wider setting of higher inductive
types, such a property would allow us to formally prove the hub-spokes
construction correct.

In \cref{induction} we give a categorical characterisation of the
induction principle as the section principle: every algebra
fibration/display map has a section. We then go on to show that in the
presence of binary products and equalisers this principle is logically
equivalent to initiality. This proof follows what one would
intuitively do in type theory if one shows that some induction
principle implies initiality: you first show that weak initiality
holds for which we need to produce a constant algebra family (this is
the construction of products). Establishing that the resulting
morphism is unique can be done by employing the induction principle
again with the motive being the pointwise equivalence of the two
morphisms. Giving the methods then amounts to given the equaliser of
the two morphisms. Since we already know the categorical structure of
the algebras, proving that the section principle and initiality
coincide then amounts to giving constructions products and
equalisers. This approach saves us from first having to come up with
an induction principle.

Interesting to note is that while initiality is a property of a
category that requires us only to have the objects and morphisms at
hand, the section principle requires the full categorical structure,
\ie we need composition, identity morphisms and laws and
associativity. Initiality is an attractive property in the light of
working in an untruncated setting, \ie working with hom-types as
opposed to hom-sets, as we do not have to bother with the category
laws and hence do not have to deal with any further coherence
laws. However, it seems that as soon as we want to do something useful
with the algebras, we need to have all category structure and laws
available.

The second part of \cref{induction} gives a derivation of the
induction principle for quotient inductive-inductive
definitions. Since we know what display maps and sections amount to in
$\Set$ in a type theoretic sense, \ie they are type families and
dependent functions, and given that all our categories are in some way
built upon $\Set$, we can use this information to derive the type
theoretic induction principle for our quotient inductive-inductive
definitions.

In \cref{constructing} we consider the existence of inductive
definitions, \ie the existence of initial objects in the categories of
algebras. The way we set things up in \cref{describing} allows us to
give inductive definitions of which the corresponding category of
algebras does not necessarily have an initial object. As our quotient
inductive-inductive definitions subsume ordinary inductive types in
the sense of providing an endofunctor on $\Set$, we can consider the
double powerset functor which does not have an initial algebra. We
therefore need to make sure that our definitions are \emph{strictly
  positive}. There are several ways to define what strict positivity
means: we can look at the syntactic expression of the functor and
check whether the arguments do not occur to the left of an arrow. For
endofunctors on $\Set$, there is the semantic characterisation of
strictly positive functors as \emph{containers}. We give a
generalisation of these to functors from any category into $\Set$,
which allows us to express the data needed for a $\Set$-sorted
quotient inductive definition. We also present a generalisation of
this to situations where the category of sorts is a presheaf category
over $\Set$. An avenue for future work would be to try to extend this
to arbitrary sort categories and that the resulting notion agrees with
the syntactic formulation.

We give constructions of initial algebras for a class of inductive
definitions where the functors are $\omega$-cocontinuous. These
constructions can be performed inside the type theory, which gives us
the result that having natural numbers and coequalisers/quotients is
enough to be able to construct a wide range of quotient
inductive-inductive definitions. Generalising this result to other
ordinals is future work. It would be interesting to investigate
whether having W-types present, along with quotients, is enough to
construct all quotient inductive-inductive definitions which are
strictly positive, similar to deriving indexed W-types from ordinary
W-types.

In \cref{untruncated} we try to lift the restriction to sets, which
would turn our theory of quotient inductive-inductive types into one
of higher inductive-inductive types with 0- and 1-constructors. (By
the hub-spokes construction this would mean that this is enough to
express also any higher inductive type with higher constructors.) As
opposed to moving to \inftycats straight away, we move from hom-sets
to hom-types and go through all the constructions to see where we run
into coherence issues. Somewhat surprisingly issues already show up
when considering only point constructors for $\Type$ sorted
definitions, \ie 0-HITs. The category of $F$-algebras for an
endofunctor on $\Type$ is no longer a category that satisfies the
category laws strictly, unlike $\Type$ itself. Even if $F$ happens to
be a strict functor, we still do not end up with a strict
\inftycat. If we add a point constructor to this category of algebras,
we increase the level of coherence needed. Therefore the number of
coherence problems we have to deal with increases with the number of
constructors, whether they are point constructors of path
constructors. To uniformly deal with this, we need to move on to full
\inftycats. The definition of these in type theory seems to require an
internal notion of strict equality.





