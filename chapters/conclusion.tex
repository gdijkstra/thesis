\chapter{Concluding remarks}
\label{conclusion}

In this thesis we have given a formal specification of quotient
inductive-inductive definitions, intended as a stepping stone towards
a theory of higher inductive(-inductive) types. This theory has been
presented in such a way that formalising it in type theory is
straightforward.

After the first two introductory and preliminary chapters, \cref{qits}
was devoted to giving examples and intuition for quotient
inductive-inductive types, before diving into the formal
definition. We argued that even only considering higher
inductive(-inductive) types truncated to sets is already a useful
extension over ordinary inductive types. The set truncation also
entails that we have to consider only the point and first path
constructors, as anything higher will be trivial. 

In the examples of \cref{qits}, we uncovered that quotient
inductive-inductive definitions set themselves apart from ordinary
inductive definitions in the following regards:
%
\begin{itemize}
\item instead of defining a single type, we may have a collection of
  \emph{dependent sorts}
\item any constructor may \emph{refer to any previous constructor}
\item the result type of a constructor may be an \emph{equation} in
  any of the sorts, \ie we allow for \emph{path constructors}
\end{itemize}
%
Both inductive-inductive definitions as well as higher inductive types
have the property of allowing for references to previous constructors,
which prompted our investigations into a uniform treatment of both
classes of inductive definitions.

In \cref{describing} we give the formal specification of quotient
inductive- inductive definitions, which is given simultaneously with
its interpretation as categories of algebras: we characterise quotient
inductive-inductive definitions roughly as iterated dialgebras
(\cref{qiids-spec}). Dealing with dependent sorts means that ordinary
dialgebras do not suffice: the categories of dialgebras are fibred
over all sorts below the sort of the current constructor. Dealing with
this means we define the category of algebras as an equaliser category
of a category of dialgebras. The category of algebras for a path
constructor is simply the category of algebras of the previous
constructors extended with an equation on those algebras given by
natural transformations.

Since we have not focussed on our specifications of inductive
definitions being sound in the sense of the categories of algebras
having initial objects, the system can also be used to work with
equational theories in type theory. While there are plenty of
equational theories that are nicely behaved in the sense that the
corresponding category of models has an initial object, this is not
always the case. A notable example of this phenomenon is the category
of fields, which can be described with our framework, but does not
have an initial object.

As we have given a specification of quotient inductive-inductive
definitions, we ought to prove properties about them. In this thesis
we investigate several properties: the logical equivalence of
initiality and induction and the construction of initial algebras via
sequential colimits.

\Cref{induction} is devoted to the correspondence between initiality
and induction in the context of quotient inductive-inductive
definitions (\cref{initiality-section-thm}). We first give a
categorical characterisation of the induction principle as the section
principle: every algebra fibration has a section. We then go on to
show that in the presence of binary products and equalisers this
principle is logically equivalent to initiality. This proof follows
what one would intuitively do in type theory if one shows that some
induction principle implies initiality: you first show that weak
initiality holds for which we need to produce a constant algebra
family (this is the construction of products). Establishing that the
resulting morphism is unique can be done by employing the induction
principle again with the motive being the pointwise equivalence of the
two morphisms. Giving the methods then amounts to giving the equaliser
of the two morphisms. Since we already know the categorical structure
of the algebras, proving that the section principle and initiality
coincide then amounts to giving constructions of products and
equalisers. This approach saves us from first having to come up with
an induction principle.

Interesting to note is that while initiality is a property of an
object that requires us only to have the objects and morphisms of the
category at hand, the section principle requires the full categorical
structure, \ie we need composition, identity morphisms and laws and
associativity. Initiality is an attractive property in the light of
working in an untruncated setting, \ie working with hom-types as
opposed to hom-sets, as we do not have to bother with the category
laws and hence do not have to deal with any further coherence
laws. However, comparing it to the section principle requires us to
use more category structure and laws.

The second part of \cref{induction} gives a derivation of the
induction principle for quotient inductive-inductive
definitions. Since we know what display maps and sections amount to in
$\SET$ in a type theoretic sense, \ie they are type families and
dependent functions, and given that all our categories are in some way
built upon $\SET$, we can use this information to derive the type
theoretic induction principle for our quotient inductive-inductive
definitions.

In \cref{constructing} we consider the existence of inductive
definitions, \ie the existence of initial objects in the categories of
algebras. The way we set things up in \cref{describing} allows us to
give inductive definitions of which the corresponding category of
algebras does not necessarily have an initial object. As our quotient
inductive-inductive definitions subsume ordinary inductive types in
the sense of providing an endofunctor on $\SET$, we can consider the
double powerset functor which does not have an initial algebra. We
therefore need to make sure that our definitions are \emph{strictly
  positive}. 

We give constructions of initial algebras for a class of $\SET$-sorted
quotient inductive-inductive definitions where the functors are
$\omega$-cocontinuous (\cref{initial-objects-qits}). These
constructions can be performed inside the type theory, which gives us
the result that having natural numbers and coequalisers/quotients is
enough to be able to construct a wide range of quotient
inductive-inductive definitions.

In \cref{containers} we present some preliminary work on
characterising strictly positive functors for quotient
inductive-inductive definitions, by generalising containers. We give a
generalisation of these to functors from any category into $\SET$,
which allows us to express the data needed for a $\SET$-sorted
quotient inductive definition. We also present a generalisation of
this to situations where the category of sorts is a presheaf category
over $\SET$.

In \cref{untruncated} we try to lift the restriction to sets: instead
of considering hom-sets in our categories of sorts and algebras, we
broaden this to hom-\emph{types}. This would turn our theory of
quotient inductive-inductive types into one of higher
inductive-inductive types with point constructors and path
constructors that construct paths between points, \ie no higher path
constructors. As opposed to moving to \inftycats straight away, we
move from hom-sets to hom-types and go through all the constructions
to see where we run into coherence issues. Somewhat surprisingly
issues already show up when considering only point constructors for
$\TYPE$-sorted definitions. The category of $F$-algebras for an
endofunctor on $\TYPE$ is no longer a category that satisfies the
category laws strictly, unlike $\TYPE$ itself. Even if $F$ happens to
be a strict functor, we still do not end up with a strict
\inftycat. If we add a point constructor to this category of algebras,
we increase the level of coherence needed. Therefore the number of
coherence problems we have to deal with increases with the number of
constructors, whether they are point constructors of path
constructors.

\section{Future work}

\subsection{Metaprogramming and generic programming}

Given that our definition of quotient inductive-inductive types can be
formulated inside type theory, one avenue for future work would be
applying this definition to generic programming ideas. Having these
definitions as the basis of the implementation of inductive
definitions in your theory is useful when one wants to use
metaprogramming techniques to define programs abstracting over data
types. One aspect of our approach is that we stay with the idea of an
inductive definition being given as a list of constructors, as opposed
to simplifying the situation to being a code of a single
endofunctor. Staying with the list of constructors idea also means
that we could build a system for writing attribute grammars internally
without needing any external tools, allowing for aspect oriented
programming.

\subsection{Invariance of descriptions under equivalence of constructors}

An important property that should hold is that the definitions should
be invariant under equivalence of constructors. If we have two
specifications $s, s' : \specty$ with the same dependent sorts, such
that if $| \Alg_s | = | \Alg_{s'} |$, \ie all the constructors
combined of $s$ are equivalent to those of $s'$, then the initial
object of $\Alg_s$ should have an isomorphic carrier to that of
$\Alg_{s'}$. This is an important property that is used often to
reason about equivalence of inductive definitions. For example, it
implies that the definitions are invariant under reordering of
constructors.

\subsection{Generalised containers}

We have given the definition of generalised containers as a means of
describing functors into $\SET$ and presheaf categories. For
descriptions of quotient inductive-inductive definitions we need to be
able to handle functors into sort categories. Generalising the
containers to support this is an avenue of future work.

Along with this one should also establish that the usual properties of
ordinary containers hold, \ie container morphisms completely describe
natural transformations between extensions of containers.

We have noted that we can use quotient inductive types to define
functors on $\SET$ which are not representable as an ordinary
container, namely propositional truncation. It would be interesting to
see whether we can adjust the definition of container to subsume such
functors.

\subsection{Constructing initial algebras}

We have given constructions of initial algebras for $\SET$-sorted
definitions where the arguments functors were
$\omega$-cocontinuous/finitary. Future work would be to generalise
this to other ordinals as well and make the construction work for
arbitrarily sorted definitions.

One possible approach would be to have internally to the type theory a
syntax for (strictly positive) quotient inductive-inductive
definitions with ordinal annotations, so one could compute at what
ordinal the colimits stabilise.

\subsection{Generalising to higher inductive types}

The ultimate goal of this project is to have a theory of higher
inductive types. In \cref{untruncated} we have shown what the kind of
issues are we run into when trying to move our results from the
category theoretic setting, where we work only with sets, to a higher
category theoretic setting without any truncation. In the chapter we
also argue that the naive approach gets unworkable very quickly. To
adequately describe a theory of higher inductive types, one has to
turn to \inftycats. Defining \inftycats in type theory is ongoing work
and seems to require extending the type theory with an internal notion
of strict equality, which allows us to talk about definitional
equalities in the type theory itself, as well as propositional
equality \cite{Altenkirch2016,Altenkirch2016iii}.
