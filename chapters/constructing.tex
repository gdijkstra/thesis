\chapter{Constructing quotient inductive types}

\section{Strict positivity}

In \cref{from-syntax-to-functors} we have seen that not every list of
constructors can be made sense of in terms of being objects of
categories of algebras. In order for the arguments of the constructors
to describe functors, the recursive occurrences have to all be in
positive positions. 

Even if we consider endofunctors on $\Set$, its initial algberas need
not exist: the types we are talking about have to be \emph{strictly}
positive. One example of a functor that does not have an initial
algebra is the \emph{double powerset functor}
$P X \ddefeq (X \to 2) \to 2$. By Lambek's theorem the initial algebra
would give us a set $X$ such that $X = P X$. If we then apply Cantor's
theorem, we arrive at a contradiction.

\section{Initial objects via sequential colimits}


\begin{theorem}[Ad\'amek]
  \label{adamek-thm}
  Let $\Cc : \Cat$ be a category with an endofunctor
  $F : \Func{\Cc}{\Cc}$. The category of algebras $\algcat{F}$ has an
  initial object if $\Cc$ has sequential colimits and $F$ preserves
  these colimits.
\end{theorem}

\begin{proof}
  
\end{proof}

\subsection{$\Set$-sorted inductive-inductive definitions}

Before we try to tackle the general case, we will look at the
simplified setting of $\Set$-sorted inductive-inductive
definitions. The categories of algebras in this case are categories of
dialgebras. As it turns out, these categories are equivalent to
ordinary categories of algebras for some endofunctor.

Observe that if we have $F : \Func{\Cc}{\Cc}$ such that $F$ and $\Cc$
satisfy the conditions of \cref{adamek-thm}, then the forgetful
functor $U : \Func{\algcat{F}}{\Cc}$ has a left adjoint:

\begin{proposition}
  Let $F : \Func{\Cc}{\Cc}$ be an endofunctor on a category
  $\Cc : \Cat$ which has sequential colimits, which $F$ preserves. The
  forgetful functor $U : \Func{algcat{F}}{\Cc}$ has a left adjoint.
\end{proposition}

\begin{proof}
  The left adjoint is defined pointwise: for $A : \Cc$, $L A$ is
  defined as the initial algebra of the functor $\lambda X . A + FX$.
  This functor again satisfies the conditions of \cref{adamek-thm}, so
  these initial algebras exist.

  We then need to check whether 
  
  
\end{proof}

\begin{proposition}
  Let $\Cc$ be a category with sequential colimits and
  $F : \Func{\Cc}{\Cc}$ be a functor that preserves them. The category
  $\algcat{F}$ has sequential colimits.
\end{proposition}

\begin{proof}
  
\end{proof}

\begin{proposition}
  Let $s : \specty$ be a specification of a $\Set$-sorted
  inductive-inductive definition. If every arguments functor $F_i$
  preserves sequential colimits, the forgetful functor
  $U : \Func{\Alg_s}{\Set}$ has a left adjoint.
\end{proposition}

\begin{proof}
  
\end{proof}

\begin{proposition}
  Let $s : \specty$ be a specification of a
  $\Set$-sorted inductive-inductive definition. If every arguments
  functor $F_i$ preserves sequential colimits, then
  $\Alg_s$ has an initial object.
\end{proposition}

\begin{proof}
  The category $\Set$ has an initial object $\initty$. There exists an
  adjunction $L \dashv U : \Alg_s$ where $U$ is the
  forgetful functor. Left adjoints preserve colimits, hence
  $L\ \initty$ is initial in $\Alg_s$.
\end{proof}

\subsection{Sort categories}

Suppose we have a sort specification $\Ss : \sortsty$ giving rise to
the following chain of sort categories:
$$
\xymatrix{
\termcat &S_0 \ar[l]_{t_0} &S_1 \ar[l]_{t_1} &\hdots \ar[l]_{t_2} &S_n \ar[l]_{t_n}
}
$$

Note that the forgetful functor $\Func{\Fam}{\Set}$ has a left
adjoint: the truth functor that maps a set $X$ to the family
$\lambda x . \unitty : X \to \Set$. This generalises to any functor in
the abovementioned chain of sort categories:

\begin{proposition}
Every forgetful functor $t_i : \Func{S_{i+1}}{S_i}$ has a left adjoint
\end{proposition}

\begin{proof}
  We define the functor $u_i : \Func{S_i}{S_{i+1}}$ as follows:
  \begin{itemize}
  \item on objects: $u_i X \ddefeq (X , \lambda x . \unitty)$
  \item on morphisms: $u_i f \ddefeq (f , \lambda a\ x . x)$
  \end{itemize}
  We then have to check whether we have for any $X : | S_i |$ and
  $(Y,Q) : | S_{i+1} |$ that
  $$
  S_{i+1}(u_i\ X , (Y,Q)) = S_i(X,Y)
  $$
  which we can show by simple equational reasoning:
  \begin{align*}
      &&S_{i+1}(u_i\ X , (Y,Q)) \\
    &=& S_{i+1}((X,\lambda x . \unitty), (Y,Q)) \\
    &=& (f : S_i(X,Y) \times (g : (a : R_{i+1} X) \to \unitty \to \unitty) \\
    &=& S_i(X,Y)
  \end{align*}

\end{proof}

\subsection{Categories of algebras}

\subsubsection{0-constructors}

\subsubsection{1-constructors}
