\chapter{Describing inductive definitions}

\section{Algebraic semantics}

\section{$\Set$-sorted inductive-inductive definitions}

We have mentioned that one of the defining features of the inductive
definitions we want to be able to handle, is any constructor may refer
to any of its previous constructors. Some examples we have seen of
this phenomenon are:
%
\begin{itemize}
\item the constructor $\intseg : \intzero = \intone$ of the interval
  type, $\intty$, which refers to both its previous constructor,
\item the constructor
  $\ttpi : (\Gamma : \ttconty) (A : \tttyty\ \Gamma) \to \tttyty\
  (\ttcons{\Gamma}{A}) \to \tttyty\ \Gamma$,
  which refers to the previous constructor
  $\ttcons{\_}{\_} : \Gamma : \ttconty) \to \tttyty\ \Gamma \to
  \ttconty$.
\end{itemize}

Describing inductive definitions with such constructors is not
something we can do with a single endofunctor on an appropriately
chosen category. Instead of having one functor, we need a functor for
every constructor. The domain of the $n$-th functor will then be the
``category of algebras of all previous constructors''. 

\begin{example}
  \label{silly-type}
  To illustrate this idea, we consider the following example (which we
  can easily show to be equivalent to the booleans):
  % 
  \begin{datatype}{\Tty}{\Type}
    \constr{\Ta}{\unitty \to \Tty} \\
    \constr{\Tb}{(t : \Tty) \times \Ta\ \unitval = t \to \Tty}
  \end{datatype}
  % 
\end{example}

The first constructor $\Ta$ can be described as being an $F_0$-algebra
structure on $\Tty$, with $F_0 : \Type \to \Type$ being defined as
$F_0 X \ddefeq \unitty$. The arguments of $\Tb$ are described by the
functor $F_1 : \algcat{F_0} \to \Type$ with:
$$
F_1 (X,a) \ddefeq (x : X) \times a\ \unitval = x
$$
Since $F_1$ is not an endofunctor, $\Tb$ cannot be described as an
$F_1$-algebra structure. However, $\algcat{F_0}$ comes with a
forgetful functor $U_0 : \algcat{U_0} \to \Type$ which gives us the
carrier of the algebra. We therefore have that
$\Tb : F_1(\Tty,\Ta) \to U_0(\Tty,\Ta)$, \ie $\Tb$ is an
$(F_1,U_0)$-\emph{dialgebra}~\cite{Hagino1987} structure on
$(\Tty,\Ta) : \algcat{F_0}$.
%
\begin{definition}
  \label{dialg}
  Let $F, G : \Cc \to \Dd$ be functors. The category $\dialgcat{F}{G}$ has
  as objects pairs $(X, \theta)$ where $X : |\Cc|$ and $\theta : F X \to
  G X$. A morphism from $(X, \theta)$ to $(Y, \rho)$ is a morphism $f : X
  \to Y$ in $\Cc$ such that the following commutes:
  $$
  \xymatrix{
  F X \ar[r]^{\theta} \ar[d]_{F f} &G X \ar[d]^{G f} \\
  G X \ar[r]^{\rho}  &G Y
  }
  $$
\end{definition}
%

As $\specty$ is essentially a list type, we will use list notation for
its values, \eg we denote $\specsnoc\ (\specsnoc\ F_0\ \specnil)\ F_1$
as $[F_0,F_1]$.

Any dialgebra category $\dialgcat{F}{G}$ comes with a forgetful
functor $U : \dialgcat{F}{G} \to \Cc$ which projects out the carrier
of the algebra. This means in particular that we have a forgetful
functor $U_1 : \dialgcat{F_1}{U_0} \to \algcat{F_0}$. If we were to
add a third and a fourth constructor, we would have to define a
functors $F_2 : \dialgcat{F_1}{U_0} \to \Type$ and
$F_3 : \dialgcat{F_2}{U_0 \circ U_1} \to \Type$ to describe
their arguments. The objects we are interested in are \emph{iterated}
dialgebras. For the situation described above we have:
$$
\xymatrix{
\Type 
\ar[d]_{F_0}
&\algcat{F_0} 
\ar[d]_{F_1}
\ar[l]_{U_0}
&\dialgcat{F_1}{U_0}
\ar[d]_{F_2}
\ar[l]_{U_1}
&\dialgcat{F_2}{U_0 \circ U_1}
\ar[d]_{F_3}
\ar[l]_{U_2}
&\dialgcat{F_3}{U_0 \circ U_1 \circ U_2}
\ar[l]_{U_3}
\\
\Type
&\Type
&\Type
&\Type
&
}
$$

Note that we build the category of dialgebras over another category of
dialgebras, \ie we are working with \emph{iterated} dialgebras. All of
the categories of dialgebras have a forgetful functor into $\Type$. We
can formalise the concept $\Type$-sorted inductive-inductive
definitions as a inductive-recursive type:
%
\begin{definition}
  \label{type-sorted-spec}
  The specification of a $\Type$-sorted inductive-inductive definition
  and its category of algebras and underlying carrier functor is given
  by the following inductive-recursive
  definition of
  %
  \begin{sorts}
    \sortnamety{\specty}{\Type} \\
    \functy{\Alg}{\specty \to \Cat} \\
    \functy{\Und}{(s : \specty) \to \Func{\Alg_s}{\Type}}
  \end{sorts}
  %
  where $\specty$ is inductively generated by
  %
  \begin{datatype}{\specty}{}
    \constr{\specnil}{\specty} \\
    \constr{\specsnoc}{(s : \specty) \to \Func{\Alg_s}{\Type} \to \specty}
  \end{datatype}
  %
  and $\Alg$ and $\Und$ are defined by
  %
  \begin{align*}
    \Alg\ \specnil &\ddefeq \Type \\
    \Alg\ (\specsnoc\ s\ F) &\ddefeq \dialgcat{F}{U_s} \\
    \\
    \Und\ \specnil &\ddefeq\ \id \\
    \Und\ (\specsnoc\ s\ F) &\ddefeq U_s \circ V
  \end{align*}
  %
  where $V$ is the forgetful functor that gives the underlying object of
  a dialgebra.
\end{definition}

\begin{example}
  The example in \cref{silly-type} can be represented by the
  specification $[F_0, F_1]$ where $F_0$ and $F_1$ are the same
  functors as before, as we have that $\Type \defeq \Alg_{[]}$,
  $\algcat{F_0} \defeq \Alg_{[F_0]}$ and
  $\dialgcat{F_1}{U_0} \defeq \Alg_{[F_0,F_1]}$.
\end{example}

\subsection{Avoiding induction-recursion}
\label{avoiding-induction-recursion}

The type of specifications in \cref{type-sorted-spec} is given
inductive-recursively. Intuitively the type is just a snoc-list of
functors. The induction-recursion allows us to succinctly make sure
that the domain of the functors is always a category of algebras. We
can avoid induction-recursion by separately defining the snoc-list of
functors and a predicate on that list that ensures the domain of the
functors is correct:

\begin{definition}
  % 
  \begin{datatype}{\specaltty}{\Type}
    \constr{\specaltnil}{\specaltty} \\
    \constr{\specaltsnoc}{\specaltty \to (\Cc : \Cat)\ (\Func{\Cc}{\Type}) \to \specaltty}
  \end{datatype}
  % 
  On this type, we define a predicate mutually with its interpretation
  function $\Algalt$ with forgetful functor $\Undalt$:
  %
  \begin{sorts}
    \functy{\specaltiscorrect}{\specaltty \to \Type} \\
    \functy{\Algalt}{(s : \specaltty) \times (\specaltiscorrect\ s) \to \Cat} \\
    \functy{\Undalt}{(s : \specaltty) \times (p : \specaltiscorrect\ s) \to \Func{\Algalt_{(s,p)}}{\Type}}
  \end{sorts}
  %
  where
  %
  \begin{align*}
    \specaltiscorrect\ \specaltnil &\ddefeq \unitty \\
    \specaltiscorrect\ (\specaltsnoc\ s\ \Cc\ F) &\ddefeq (p : \specaltiscorrect) \times (\Cc = \Algalt_{(s,p)})
  \end{align*}
  %
  The definitions of $\Alg'$ and $\Und'$ are similar to the previous
  definitions: we can pattern match on the equality proofs we get from
  $\specaltiscorrect$ and then use the previous definitions.
\end{definition}

\begin{remark}
  The mutual definition of $\specaltiscorrect$, $\Alg'$ and $\Und'$
  can be avoided by combining all three definitions into one function
  with all its arguments and result types combined in a big
  $\Sigma$-type.
\end{remark}

\begin{proposition}
  The types $\specty$ and $(s : \specaltty) \times \specaltiscorrect\ s$ are equivalent.
\end{proposition}

\begin{proof}
  This is a straightforward proof by induction on the types involved.
\end{proof}

\section{Dependent sorts}

Describing how a sort depends the previous sorts can be done by
providing a functor of the previous category of sorts into $\Type$. A
complete description is then a snoc-list of functors, which can be
formalised inductive-recursively together with the function that
interprets the list as a category.
% 
\begin{definition}
  The specification of sorts and their interpretation as a
  categories is given by the following inductive-recursive
  definition
  % 
  \begin{sorts}
    \sortnamety{\sortsty}{\Type} \\
    \functy{\SortCat{\_}}{\sortsty \to \Cat}
  \end{sorts}
  where $\sortsty$ is inductively generated by
  %
  \begin{datatype}{\sortsty}{}
    \constr{\sortsnil}{\sortsty} \\
    \constr{\sortssnoc}{(\Ss : \sortsty) \to (\Func{\SortCat{\Ss}}{\Type}) \to \sortsty}
  \end{datatype}
  % 
  with $\SortCat{\sortsnil}$ defined as the terminal category $\termcat$, and given
  $\Ss : \sortsty$ and $R : \Func{\SortCat{\Ss}}{\Type}$,
  the category $\SortCat{\sortssnoc\ \Ss\ R}$ has:
  % 
  \begin{itemize}
  \item objects: $(X : | \SortCat{\Ss} |) \times (R X \to \Type))$,
  \item morphisms $(X,Z) \to (Y,W)$ consists of
    \begin{itemize}
    \item a morphism $f : \SortCat{\Ss}(X,Y)$
    \item a dependent function
      $g : (x : R X) \to Y\ x \to W\ (R\ f\ x)$.
    \end{itemize}
  \end{itemize}
\end{definition}

\begin{remark}
  The use of induction-recursion can be avoided here as well, using
  the same techniques as in \cref{avoiding-induction-recursion}
\end{remark}

\begin{example}
  The sort of an ordinary inductive definition can be represented by
  the list $[R_0]$ (\ie $\sortssnoc\ \sortsnil\ R_0$) where
  $R_0 : \Func{\termcat}{\Type}$ is defined as the constant functor
  $R_0\ x \ddefeq \unitty$. The resulting category $\SortCat{R_0}$ has
  objects $(\unitval : \unitty) \times (A : \unitty \to \Type)$ and a
  morphism $(x, A) \to (y, B)$ is given by, since trivially
  $x = y = \unitval$, a trivial morphism $\unitty \to \unitty$
  together with a function $f\ x : A\ x \to B\ x$ for every $x : 1$. In
  other words, this category is equivalent to the category $\Type$.
\end{example}

In the above example it may seem superfluous to have the empty list
interpreted as the terminal category and not as $\Type$. However, this
choice allows us to have the first sort be indexed by some other type,
\eg $\natty$. Hence a definition of the vectors would have as sort
specification the list $[R_0]$ with $R_0\ x \ddefeq \natty$.

\begin{example}
  The sort of the context and types example $(\ttconty, \tttyty)$ can be
  represented by the list $[R_0, R_1]$ with
  \begin{itemize}
  \item $R_0 : \Func{\termcat}{\Type}$, $R_0\ x \ddefeq \unitty$
  \item $R_1 : \Func{\SortCat{R_0}}{\Type}$, $R_0 (x, A) \ddefeq A\ x$
  \end{itemize}
  The category $\SortCat{R_0,R_1}$ has objects
  $(x : \unitty) \times (A : \unitty \to \Type) \times (B : A\ x \to
  \Type)$.
  We see that this category is equivalent to the category $\Fam$ of
  families of sets.
\end{example}

\begin{example}
  \label{rel-sorts}
  Similarly, the category $\Rel$ can be represented by the list
  $[R_0, R_1]$ with
  \begin{itemize}
  \item $R_0 : \Func{\termcat}{\Type}$, $R_0\ x \ddefeq 1$
  \item $R_1 : \Func{\SortCat{R_0}}{\Type}$, $R_0 (x, A) \ddefeq A\ x \times A\ x$
  \end{itemize}
  We see that $\SortCat{R_0,R_1}$ is equivalent to the category with
  objects $(X : \Type, R : X \to X \to \Type)$ and morphisms
  $(X,R) \to (Y,S)$ given by $f : X \to Y$ together with
  $g : (x\ y: X) \to R\ x\ y \to S\ (f\ x)\ (f\ y)$.
\end{example}

\subsection{Sort membership}

A specification $\Ss = [R_0, \ldots, R_n] : \sortsty$ defines a chain of categories
\[
\xymatrix{
1 &S_0 \ar[l]_{t_0} &S_1 \ar[l]_{t_1} &\hdots \ar[l]_{t_2} &S_n \ar[l]_{t_i}
}
\]
where $n$ is the number of functors in the list and
$S_i = \SortCat{R_0, \ldots, R_i}$ is the category of sorts truncated
to the first $i$ elements. Every $t_i$ is the forgetful functor that
maps $(X,Y)$ to $X$.

\begin{example}
In the case of $\Rel$, we get the sequence
\[
\xymatrix{
1 &\Type \ar[l]_{t_0} &\Rel \ar[l]_{t_1}
}
\]
\end{example}

When giving a quotient inductive-inductive definition, we start out by
defining its sorts. When specifying a constructor, we need to say what
sort we want to construct points or paths in. To this end, we can
define a membership relation
\[
  \_ \in \_ : \Cat \to \sortsty \to \Type
\]
where $\Cc \in \Ss$ means that $\Cc$ appears in the chain
${1 \leftarrow S_0 \leftarrow \ldots \leftarrow S_n}$ corresponding to
$\Ss$, \ie $\Cc$ is either $1$, or $S_i$ for some $0 \leq i \leq
n$.
For a quotient inductive-inductive definition with sorts
$\Ss : \sortsty$, specifying the sort for a constructor is done by
giving a category $\Aa : \Cat$ along with a proof of $\Aa \in \Ss$.

\subsection{Makkai's dependent sorts}

\section{Categories of algebras}

\subsection{A $\Rel$-sorted quotient inductive-inductive type}

As a warm-up, in this section we will give an example of how the
specification and the categories of algebras look like for the
following quotient inductive-inductive definition:
%
\begin{sorts}
  \sortnamety{\Aty}{\Set} \\
  \sortnamety{\Bty}{\Aty \to \Aty \to \Set}
\end{sorts}
%
\begin{datatype}{\Aty}{}
  \constr{\Aco}{\Aty} \\
  \constr{\Aci}{\Aty}
\end{datatype}
%
\begin{datatype}{\Bty}{}
  \constr{\Bcii}{\natty \to \Bty\ \Aco\ \Aci} \\
  \constr{\Bciii}{(n : \natty) \to \Bcii\ n = \Bcii\ (n+1)}
\end{datatype}

The sorts of the definition are
$(A : \Set) \times (B : A \to A \to \Set)$: the category of sorts is
the category $\Rel$. In \cref{rel-sorts} we saw how this can be
represented as a list of functors. This category is also the first
category of algebras, \ie algebras with no constructors, and will as
such also be referred to as $\Alg_0$.

\subsubsection{Specifying constructors}
The first constructor $\Aco$ has no arguments and is of sort
$\Aty : \Set$. Its arguments can be described by the functor
$F_0 : \Func{\Rel}{\Set}$ with $F_0\ (X,R) \ddefeq \unitty$. The
category of algebras for the first constructor is then
$\dialgcat{F_0}{t_1}$, where $t_1 : \Func{\Rel}{\Set}$ is the
forgetful functor. Strictly speaking, an $(F_0,t_1)$-dialgebra
structure on a relation $(X,R)$ is a function $1 \to X$, but for this
example we will work with the equivalent definition
$| \Alg_1 | \equiv (X : \Set) \times (R : X \to X \to \Set) \times
(\theta_0 : X)$.
Morphisms $(X,R,\theta_0) \to (Y,S,\rho_0)$ consist of:
%
\begin{itemize}
\item $f : X \to Y$
\item $g : (x\ y : X) \to R\ x\ y \to S\ (f\ x)\ (f\ y)$
\item $f_0 : f\ \theta_0 = \rho_0$
\end{itemize}
%
We see that we get a morphism in $\Rel$ along with a computation rule
that tells us that the morphism in $\Rel$ preserves the
$(F_0,t_1)$-dialgebra structure.
%
Note that $F_0$ is a functor from $\Rel$ to $\Set$ and is not an
endofunctor on $\Set$: the constructor may refer to elements of fibres
of the relation $X \to X \to \Set$ being defined. The category
$\Alg_1$ also comes with a forgetful functor $U_1 : \Func{\Alg_1}{\Alg_0}$
defined by $U_1 ((X,R),\theta_0) \ddefeq (X,R)$ --- in fact, every
category of algebras $\Alg_{i+1}$ has a similarly defined forgetful
functor $U_{i+1} : \Func{\Alg_{i+1}}{\Alg_0$}.

For the second constructor $\Aci$, the specification is largely
similar: it is given by the functor $F_1 : \Func{\Alg_1}{\Set}$
defined by $F_1\ (X, R, \theta_0) \ddefeq \unitty$.

The third constructor $\Bcii$ maps into a different sort, hence its
definition will be slightly different. We want the resulting category
$\Alg_3$ of algebras for constructors $\Aco, \Aci, \Bcii$ to have
objects
$((X,R) : \Rel) \times (\theta_0 : X) \times (\theta_1 : X) \times
(\theta_2 : \natty \to R\ \theta_0\ \theta_1)$,
and we want morphisms
$((X,R),\theta_0,\theta_1,\theta_2) \to ((Y,S),\rho_0,\rho_1,\rho_2)$
to consist of
$(f,g,f_0,f_1) :
\Alg_2(((X,R),\theta_0,\theta_1),((Y,S),\rho_0,\rho_1))$
together with an equality
$$
g_2 : (n : \natty)  \to \pathover{S}{(f_0,f_1)}{g\ \theta_0\ \theta_1\ (\theta_2\ n)}{\rho_2\ n}
$$
%WAS: g_2 : (x\ y: X) (n : \natty)  \to g\ x\ y\ (\theta_2\ n) = \rho_2\ n
% -- CHECK new version is correct!
Note how we have to use the equalities $f_0 : f\ \theta_0 = \rho_0$ and
$f_1 : f\ \theta_1 = \rho_1$ to reconcile the types of
$g\ \theta_1\ \theta_2\ (\theta_2\ n) : S\ (f\ \theta_0)\ (f\
\theta_1)$ and $\rho_2 : S\ \rho_0\ \rho_1$.
%
Realising the above directly as a dialgebra category is a bit
tricky. However, we can rewrite the type of
$\theta_2 : \natty \to R\ \theta_0\ \theta_1$ to the equivalent ``Henry
Ford''-style type (you can have any $(x, y)$ you want, as long as it's
$(\theta_0, \theta_1)$)\
\[
\theta_2' : (x\ y : X) \to \big((x = \theta_0) \times (y = \theta_1) \times \natty\big) \to R\ x\ y 
\]
 %
and define the functor $F_2 : Alg_2 \to \Rel$ by
$F_2 ((X,Y),\theta_0,\theta_1) \ddefeq (X , \lambda x\ y . (x =
\theta_0) \times (y = \theta_1) \times \natty)$
in order to see that $\theta_2'$ is a $(F_2,U_2)$-dialgebra. The
objects in this category give us ``too much'': such a dialgebra gives
us a morphism $F_2((X,Y),\theta_0,\theta_1) \to (X,Y)$ in $\Rel$, so
we also get a function $X \to X$. We can solve this problem by adding
an equation that the function $X \to X$ need be the identity. $\Alg_3$
is therefore not a category of dialgebras directly, but an equaliser
of one, as we will elucidate later on in some section.

The fourth constructor $\Bciii$ is a path constructor. Hence, not only
do we need to supply a functor $F_3 : \Func{\Alg_3}{\Alg_0}$ to
specify the arguments, but we also need to specify the endpoints of
the path. Just as for $\Bcii$, we will first need to rewrite the type
of the constructor. We denote again by $\theta_2'$ the Henry
Ford-version of $\theta_2$.  We then have that the type
$(n : \natty) \to \theta_2\ n = \theta_2\ (n+ 1)$ is equivalent to the
type
%
\begin{multline*}
(x\ y : X) \to \big((p : x =  \theta_0) \times (q : y = \theta_1) \times (n : \natty)\big) \\
\ \ \ \ \to \theta_2'\ x\ y\ p\ q\ n = \theta_2'\ x\ y\ p\ q\ (n+1)
\end{multline*}
%
The endpoints can then be specified as natural transformations
$l, r : \Nat{F_3}{U_3}$, where $F_3 : \Func{\Alg_3}{\Rel}$ is defined in a
way similar to $F_2$. Given an algebra $A : \Alg_3$, $l_A$ and $r_A$
both define a morphism in
$\Rel$. $l_{((X,Y),\theta_0,\theta_1,\theta_2)}$ and
$r_{((X,Y),\theta_0,\theta_1,\theta_2)}$ are defined as
$(\id_X, l^1_{((X,Y),\theta_0,\theta_1,\theta_2)})$ and
$(\id_X,r^1_{((X,Y),\theta_0,\theta_1,\theta_2)})$ respectively, with
%
\begin{align*}
&l^1_{((X,Y),\theta_0,\theta_1,\theta_2)}\ a\ b\ (p, q, n) \ddefeq \theta_2'\ a\ b\ p\ q\ n \\
&r^1_{((X,Y),\theta_0,\theta_1,\theta_2)}\ a\ b\ (p, q, n) \ddefeq \theta_2'\ a\ b\ p\ q\ (n+1)
\end{align*}
%
By function extensionality, we can then say the category of algebras
$\Alg_3$ has objects $(X : | \Alg_2 |) \times (\theta_3 : l_X = r_X)$,
\ie it is a equaliser category. The morphisms are just morphisms in
$\Alg_2$ with no extra structure. For higher inductive types, one
usually expects a path computation rule for any path constructor, but
as we are working with sets, equalities between paths are trivial.

\subsection{0-constructors}

\subsection{1-constructors}


\section{From syntax to functors}

