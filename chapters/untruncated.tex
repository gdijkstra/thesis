\chapter{Moving to an untruncated setting}
\label{untruncated}

In the previous chapters, we have mostly worked with sets. In this
chapter we show what kind of issues one encounters if we work in an
untruncated setting instead, motivating why we went with the choice to
work in the set truncated setting in the first place. If we want to
generalise the theory to higher inductive types, we have to lift this
restriction and somehow deal with these issues.

The place where we had to ensure that certain types were sets, was in
the definition of category: we worked with \emph{sets} of
morphisms. This means that the category laws are \emph{propositions}
and saves us from having to worry too much about reasoning about
equality of morphisms: any two such proofs will be equal.

If we have categories with hom-\emph{types}, then we have no
guarantees that the category laws interact nicely. For example, if we
have four composable morphisms:
$$
\xymatrix{
  X \ar[r]^{f} &Y \ar[r]^{g} &Z \ar[r]^{h} &W \ar[r]^{i} &V
}
$$
If we want to show that the following equation holds:
$$
((i \circ h) \circ g) \circ f = i \circ (h \circ (g \circ f))
$$
then we have a choice in what order we apply the category laws. There
is a priori no guarantee that these choices yield equal proofs: we
need to add this assumption as an extra \emph{coherence condition}. In
the case of associativity interacting with itself, it is as follows:

\begin{definition}[Coherence condition for associativity]
  The witness of associativity
  $$
  \assoc : (h : \Cc(Z,W)) (g : \Cc(Y,Z)) (f : \Cc(X,Y)) \to ((h \circ g) \circ f) = (h \circ (g \circ f))
  $$
  is \emph{coherent} if for any composable arrows $i, h, g, f$ the
  following commutes:
  $$
  \xymatrix{
    &(i \circ (h \circ g)) \circ f \ar@{-}[dr]^{\assoc i (h \circ g) f} &\\
    ((i \circ h) \circ g) \circ f \ar@{-}[ur]^{(\assoc i h g) \circ f} \ar@{-}[d]_{\assoc (i \circ h) g f}  &&i \circ ((h \circ g) \circ f) \ar@{-}[d]^{i \circ \assoc h g f}\\
    (i \circ h) \circ (g \circ f) \ar@{-}[rr]_{\assoc i h (g \circ f)} &&i \circ (h \circ (g \circ f))
  }
  $$
\end{definition}

We can formulate similar conditions for the interactions between the
identity laws and associativity. The story does not end there however:
there is no reason to assume that the equality between equalities of
morphism is propositional. The coherence condition of associativity
needs to behave nicely in harmony with the other category laws and
coherence conditions: the coherence conditions themselves require
further coherence conditions. This phenomenon does not stop: we get an
infinite tower of coherence conditions.

Sometimes we are lucky with the concrete categories we are working
with, in that the category laws hold up to definitional equality: they
are satisfied \emph{strictly}. Examples of such categories are $\Type$
and $\Fam$. In these cases, as the category laws themselves are
satisfied trivially, any higher coherence law will also be satisfied
trivially.

For categories of algebras, this is not the case. Even if we consider
$F$-algebras with $F$ a strict endofunctor on $\Type$, $\algcat{F}$
will not satisfy the category laws strictly, with the usual
definition. In \cite{Cranch2013}, the author shows essentially that we
cannot find a definition of the category of pointed types that
satisfies the category laws strictly. Since pointed types are a
special case of $F$-algebras, we therefore cannot hope to find a nice
of $\algcat{F}$ that is strict.

The way to deal with coherences is by considering \inftycats
\cite{Camarena2013}. The usual definitions/models of \inftycats are
given by using simplicial sets. This is not a practical approach for
our purposes, as we cannot easily move between simplicial sets and
types. Another approach is to consider simplicial \emph{types}, which
is as of yet an open problem in homotopy type theory. One attempt to
solve this and give a definition of \inftycats is by extending the
type theory with an internal notion of strict equality
\cite{Altenkirch2016,Altenkirch2016iii}.

However, as we are only concerned with a finite amount of categorical
structure: we are really only interested in the category of algebras
satisfying the identity and associativity laws, we will in this
chapter see how far we get by taking a \emph{lazy} approach. We will
only add those coherence conditions we actually need to show that the
categorical laws are satisfied in all the categories of algebras.

\section{Coherence laws for functors}

Talking about the category laws in an untruncated setting also means
that we now have to worry about what it means for a functor to
preserve them. If we take for example the left identity law, we notice
that there are multiple ways to produce an equality
$F (id Y \circ f) = F f$. we can either use the left identity law of
the domain of $F$ or the one of its codomain, by first appealing to
the fact that $F$ preserves composition and identity morphisms. The
functor $F$ preserves the left identity law if these two approaches
yield the same equality:

\begin{definition}
  A functor $F : \Func{\Cc}{\Dd}$ satisfies the coherence law for the
  left identity law of $\Cc$ if the following commutes for any $f : \Cc(X,Y)$:
  $$
  \xymatrix{
    F (\id_{\Cc} Y \circ_{\Cc} f) \ar@{-}[rr]^{F (\leftid_{\Cc} f)} \ar@{-}[d]_{\Fcomp (\id_{\Cc} Y) f} &&F f \ar@{-}[d]^{\leftid_{\Dd} F f} \\
    F (\id_{\Cc} Y) \circ_{\Dd} f \ar@{-}[rr]_{\Fid Y \circ_{\Dd} f} &&\id_{\Dd} (F Y) \circ_{\Dd} f
  }
  $$  
\end{definition}

The coherence law for the right identity law is defined similarly.

\begin{definition}
  A functor $F : \Func{\Cc}{\Dd}$ satisfies the coherence law for the
  associativity law of $\Cc$ if, given three composable arrows:
  $$
  \xymatrix{
    X \ar[r]^{f} &Y \ar[r]^{g} &Z \ar[r]^{h} &W
  }
  $$
  the following commutes
  $$
  \xymatrix{
    F ((h \circ g) \circ f) \ar@{-}[rr]^{F (\assoc_{\Cc} h g f)} \ar@{-}[d]_{\Fcomp (h \circ g) f} &&F (h \circ (g \circ f)) \ar@{-}[d]^{\Fcomp h (g \circ f)} \\
    F (h \circ g) \circ F f \ar@{-}[d]_{\Fcomp h g \circ F f} &&F h \circ F (g \circ f) \ar@{-}[d]^{F h \circ \Fcomp g f} \\
    (F h \circ F g) \circ F f  \ar@{-}[rr]_{\assoc_{\Dd} Fh Fg Ff} && F h \circ (F g \circ F f)
  }
  $$
\end{definition}

\subsection{Generalised containers}

Although we are not able to internally define the type of strict
functors, there is a definable class of functors that happen to be
strict: containers on $\Type$ (see \cref{def-container}). As the
functorial action of containers are defined in terms of composition of
functions in $\Type$, the functor laws reduce to the identity and
associativity laws of $\Type$, which are satisfied strictly. As such,
containers on $\Type$ form a class of strict functors.

For generalised containers (\cref{def-gen-container}), the functors
again inherit the functor laws from the category laws of the domain of
the functor. If the generalised container maps out of a strict
category, then it defines a strict functor. However, in practice,
these categories will not be strict as they will be categories of
algebras.

\section{Sort categories}
\label{coherence-sort-categories}

The categories $\Type$ and $\Fam$ are satisfy the category laws and
their (higher) coherences definitionally. Whether a sorts category
$\SortCat{\Ss}$ given by a $\Ss : \sortsty$ depends on the functors in
the list $\Ss$. Looking at the definition of the sorts categories, if
all the functors involved satisfy the functor laws definitionally then
the resulting categories will be strict as well. This means that if we
give all the functors in $\Ss$ as generalised containers, we end up
with strict sort categories, which is quite a reasonable assumption to
make.

\section{Categories of dialgebras}

So far we have not given a precise definition of $\dialgcat{F}{G}$ for
some functors $F, G : \Cc \to \Dd$: we have not formally defined
composition and so on. In \cref{dialg}, we have defined the category
its objects and morphisms. The objects are defined in terms of objects
from $\Cc$ and morphisms from $\Dd$. Morphisms are defined in terms of
morphisms in $\Cc$ and $\Dd$ along with equalities between them. If we
are only interested in objects and morphisms of $\dialgcat{F}{G}$, we
need to know what the objects and morphisms of $\Cc$ and $\Dd$ are and
the actions of $F$ and $G$ on those. Once we go beyond morphisms, we
run into trouble.

To illustrate these issues, we will look at the definition of the
$\dialgcat{F}{G}$ in detail. 

\subsection{Identity morphisms}

Given an object $(X,\theta) : | \dialgcat{F}{G} |$, we define the
identity morphism as
$$
\id_{\dialgcat{F}{G}}\ (X,\theta) \ddefeq (\id_\Cc\ X , \id_0)
$$
with $\id_0$ defined as:
$$
\xymatrix{
G\ \id_\Cc \circ \theta
\ar@{-}[rr]^{\id_0}
\ar@{-}[d]_{\Gid \circ \theta}
&
&\theta \circ F\ \id_\Cc
\ar@{-}[d]^{\theta \circ \Fid}
\\
\id_\Dd \circ \theta
\ar@{-}[dr]_{\leftid_{\Dd}}
&
&\theta \circ \id_\Dd
\ar@{-}[dl]^{\rightid_{\Dd}}
\\
&\theta
&
}
$$

Unsurprisingly, the construction of identity morphisms relies on the
functor $F$ and $G$ preserving identity morphisms. Perhaps more
surprising is that the construction also relies on the identity
\emph{laws} of the category $\Dd$, \ie a category structure ``one
level up'' from identity morphisms.

In our cases, the codomain of functors $F$ and $G$ is always $\Type$
or some other category of sorts: it does not change with the number of
constructors.

\subsection{Composition}

Suppose we are given algebras
$(X,\theta),(Y,\rho)(Z,\zeta) : | \dialgcat{F}{G} |$ and morphisms
$(g,g_0) : (Y,\rho) \to (Z,\zeta)$ and
$(f,f_0) : (X,\theta) \to (Y,\rho)$. If we want to compose the two
morphisms, we need a way to glue the squares $g_0$ and $f_0$ together,
\ie we need an operation:
$$
\xymatrix{
 FX \ar[r]^\theta \ar[d]_{F f}                 \ar@{}[dr]|{f_0} &GX \ar[d]^{G f}="Gf"
&
&FY \ar[r]^\rho   \ar[d]_{F g}="Fg"            \ar@{}[dr]|{g_0} &GY \ar[d]^{G g}="Gg"
&
&FX \ar[r]^\theta \ar[d]_{F (g \circ f)}="Fgf" \ar@{}[dr]|{g_0 \circ_0 f_0} &GX \ar[d]^{G (g \circ f)}
\\ 
 FY \ar[r]_{\rho}  &GY
&
&FZ \ar[r]_{\zeta} &GZ
&
&FZ \ar[r]_{\zeta} &GZ
\ar "Gf";"Fg"
\ar "Gg";"Fgf"
}
$$
The (vertical) composition of the squares $g_0 \circ_0 f_0$ is defined as the composite:
$$
\xymatrix{
G\ (g \circ f) \circ \theta
\ar@{-}[r]^{g_0 \circ_0 f_0}
\ar@{-}[d]_{\Gcomp g f \circ \theta}
&\zeta \circ F(g \circ f)
\ar@{-}[d]^{\zeta \circ \Fcomp g f}
\\
(G\ g \circ G\ f) \circ \theta
\ar@{-}[d]_{\assoc_{\Dd}}
&\zeta \circ (F\ g \circ F\ f)
\ar@{-}[d]^{\assoc_{\Dd}}
\\
G\ g \circ (G\ f \circ \theta)
\ar@{-}[d]_{G g \circ f_0}
&(\zeta \circ F\ g) \circ F\ f
\ar@{-}[d]^{g_0 \circ F f}
\\
G\ g \circ (\rho \circ F\ f)
\ar@{-}[r]_{\assoc_{\Dd}}
&(G\ g \circ \rho) \circ F\ f
}
$$
Composition of dialgebra morphisms is then:
$$
(g,g_0) \circ (f,f_0) \ddefeq (g \circ f , g_0 \circ_0 f_0)
$$
As with identity morphisms, we notice that we need to appeal to the
functors preserving the same kind of structure we are defining here:
they need to preserve composition. We also need categorical structure
one level up from composition from the category $\Dd$, namely
associativity.

Looking at this definition of composition, we notice that even when
$\Cc$ and $\Dd$ are strict categories, with $F$ and $G$ being strict
functors, \eg when we consider $F$-algebras on $\Type$ with $F$ given
as a container, composition will not be strictly associative.

\subsection{Category laws}

If we want to talk about the identity laws in a category of
dialgebras, we need to know what equality between dialgebra morphisms
looks like. We can characterise it as follows:

\begin{proposition}
  \label{morphism-equality}
  Let $(f,f_0),(g,g_0)$ be two dialgebra morphisms
  $(X,\theta) \to (Y,\rho)$ in $\dialgcat{F}{G}$, then we have the
  following equivalence of equalities:
  \begin{align*}
    ((f,f_0) = (g,g_0)) =&\ (p : f = g)\\
    \times&\ (p_0 : 
              \xymatrix{
              G\ f \circ \theta
              \ar@{-}[r]^{f_0}
              \ar@{-}[d]_{G p \circ \theta}
              &\rho \circ F\ f
              \ar@{-}[d]^{\rho \circ F p}
              \\
              G\ g \circ \theta
              \ar@{-}[r]_{g_0}
              &\rho \circ F\ g
              })
  \end{align*}
\end{proposition}

\begin{proof}
  Using the fact that an equality of dependent pairs is a dependent pair of equalities, we get that
  $$
  ((f,f_0) = (g,g_0)) = (p : f = g) \times (p_0 : \pathover{\lambda h . G h \circ \theta = \rho \circ F h}{p}{f_0}{g_0}
  $$
  Having a path $p_0$ over a family of equalities is equivalent to the
  square by the usual reasoning.
\end{proof}

To give a witness for the left identity law, we need to show given:
\begin{itemize}
\item objects $(X,\theta), (Y,\rho) : |\dialgcat{F}{G}|$
\item with a morphism $(f,f_0) : (X,\theta) \to (Y,\rho)$
\end{itemize}
that $\id_{\dialgcat{F}{G}}\ (Y,\rho) \circ_{\dialgcat{F}{G}} (f,f_0) = (f,f_0)$. Unfolding definitions this reduces to having to show that:
$$
(\id_{\Cc}\ Y \circ_{\Cc} f , \id_0\ \rho \circ_0 f_0) = (f,f_0)
$$
Applying \cref{morphism-equality} this is the same as giving a proof $p : \id_{\Cc}\ Y \circ f = f$ along with a square:
$$
\xymatrix{
  G\ (\id_{\Cc}\ Y \circ f) \circ \theta
  \ar@{-}[rr]^{\id_0 \rho \circ_0 f_0}
  \ar@{-}[d]_{G p \circ \theta}
  &&\rho \circ F\ (\id_{\Cc}\ Y \circ f)
  \ar@{-}[d]^{\rho \circ F p}
  \\
  G\ f \circ \theta
  \ar@{-}[rr]_{f_0}
  &&\rho \circ F\ f
}
$$
We will work this out in detail, assuming the category $\Dd$ is
strict. This assumption is reasonable as per the reasons given in
\cref{coherence-sort-categories}.

For $p$ we can fill in $\leftid_{\Cc}\ f$. Furthermore we know that,
assuming the functors $F$ and $G$ preserve the left identity laws:
$$
F (\leftid_{\Cc}\ f) = \trans{\Fcomp\ (\id_{\Cc}\ Y)\ f}{\Fid\ Y \circ Ff}
$$
and similarly for $G$. The diagram we end up with is the one shown in
\cref{left-identity-calculation}, where the double lines indicate a
$\refl$ path. What we end up with is having to establish that the
following square commutes:
$$
\xymatrix{
G (\id_{\Cc} Y) \circ Gf \circ \theta
\ar@{-}[rr]^{G (\id_{\Cc} Y) \circ f_0}
\ar@{-}[d]_{\Gid Y \circ G f \circ \theta}
&&G (\id_{\Cc} Y) \circ \rho \circ Ff
\ar@{-}[d]^{\Gid Y \circ \rho \circ Ff}
\\
Gf \circ \theta
\ar@{-}[rr]_{f_0}
&&\rho \circ Ff
}
$$
Since we have $f_0 = \id_{\Dd} GY \circ f_0$, the above square follows
from the naturality property enjoyed by homotopies.

As we see, even when we simplify the situation by assuming that $\Dd$
is strict, we end up with a rather involved calculation.  Working out
the associativity law in detail is even more strenuous. It seems that
we can show that the category $\dialgcat{F}{G}$ has all the category
laws that are preserved by the functors $F$ and $G$.

\begin{figure}
  \centering
  $$
\xymatrix@!C@!R{
G f \circ \theta % C
\ar@{-}[r]^{\Gid Y \circ Gf \circ \theta}
\ar@{-}[ddddd]_{f_0}
&
G (\id_{\Cc} Y) \circ Gf \circ \theta % B
\ar@{-}[r]^{\Gcomp (\id_{\Cc} Y) f \circ \theta}
\ar@{=}[rd]
&G (\id_{\Cc} Y \circ f) \circ \theta % A
\ar@{-}[d]^{\Gcomp (\id_{\Cc} Y) f \circ \theta}
\\
&&G (\id_{\Cc} Y) \circ Gf \circ \theta % D
\ar@{-}[d]^{G (\id_{\Cc} Y) \circ f_0}
\\
&&G (\id_{\Cc} Y) \circ \rho \circ Ff % E
\ar@{-}[d]^{\Gid Y \circ \rho \circ Ff}
\\
&&\rho \circ Ff % F
\ar@{-}[d]^{\rho \circ \Fid Y \circ Ff}
\\
&&\rho \circ F (\id_{\Cc} Y) \circ Ff % G
\ar@{-}[d]^{\rho \circ \Fcomp (\id_{\Cc} Y) f}
\\
\rho \circ Ff % J
\ar@{-}[r]_{\rho \circ \Fid Y \circ Ff}
\ar@{=}[uurr]
&\rho \circ F (\id_{\Cc}) \circ Ff % I
\ar@{-}[r]_{\rho \circ \Fcomp (\id_{\Cc} Y) f}
\ar@{=}[ur]
&\rho \circ F (\id_{\Cc} Y \circ f) % H
}
$$

\caption{Establishing the left identity law}
\label{left-identity-calculation}
\end{figure}

\section{Untruncated $\Set$-sorted inductive-inductive definitions}

In the previous section we have sketched how the category laws can be
established in the untruncated dialgebra categories, assuming the
codomain of the functors is a strict category. This means that it
explains what we need from of a $\Set$-sorted inductive-inductive
definition to be able to show that the resulting category of algebras
does in fact satisfy the category laws: all the arguments functors
have to preserve the category laws.

While it seems that the coherence data is constant in the amount of
constructors, in practice we see that this is not the case. To show
that a functor preserves category laws, we may need more structure
from the domain category. For generalised containers, we see that to
show it preserves identity morphisms, we need to have identity laws in
the domain category, \ie we need coherence ``one level up''. To show
that it preserves the identity laws, we need the higher coherence
condition for identity laws from the domain category. This means that
the amount of coherence levels needed stacks up every time we add a
constructor to the definition.

What is somewhat surprising about this, is that we have not even
considered path constructors yet. The coherence issues grow in the
number of constructors no matter whether they are point or path
constructors.

\section{Path constructors and their computation rules}

In the set truncated setting we could ignore the computation rules for
path constructors, as all paths between equalities were trivial. In
the untruncated setting they need to be accounted for. Suppose we have
a category of algebras $\Cc$ with forgetful functor
$U : \Func{\Cc}{\Set}$. A path constructor defined on this category is
given by a functor $F : \Func{\Cc}{\Set}$ along with two natural
transformations $l,r : \Nat{F}{U}$. Let $X, Y : | \Cc |$, with
$\theta : \ell_X = r_X$ and $\rho : \ell_Y = r_Y$. A morphism $f : X \to Y$
preserves the algebra structures $\theta$ and $\rho$, if ``applying''
the equality $\theta$ first and then the morphism yields the same
equality as first applying the morphism and then the equality
$\rho$. We would like to say:
$$
U f \circ \theta = \rho \circ F f
$$
but this does not type check as we have
$U f \circ \theta : U f \circ \ell_X = U f \circ r_X$ and
$\rho \circ Ff : \ell_Y \circ F f = r_Y \circ F f$. We have to appeal to
naturality in order for this equation to make sense. Let us denote for
the witnesses of naturality as $\ell_f : Uf \circ \ell_X = \ell_Y \circ
Ff$.
For $f$ to be an algebra morphism requires us to have a witness of the
following:
$$
\xymatrix{
U f \circ \ell_X
\ar@{-}[r]^{U f \circ \theta}
\ar@{-}[d]_{\ell_f}
&U f \circ r_X
\ar@{-}[d]^{r_f}
\\
\ell_Y \circ Ff
\ar@{-}[r]_{\rho \circ Ff}
&r_Y \circ Ff
}
$$
