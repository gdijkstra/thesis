\chapter{Moving to an untruncated setting}

\section{Sort categories}

The categories $\Type$ and $\Fam$ are satisfy the category laws and
their (higher) coherences definitionally. Whether a sorts category
$\SortCat{s}$ given by a $s : \sortsty$ depends on the functors in the
list $s$. Looking at the definition of the sorts categories, if all
the functors involved satisfy the functor laws definitionally then the
resulting categories will be strict as well.

Although we are not able to internally define the type of strict
functors, there is a definable class of functors that happen to be
strict: containers.

\begin{definition}
  A \emph{container} on $\Type$ is a type of \emph{shapes} $S : \Type$
  along with a family of \emph{positions} $P : S \to \Type$, denoted
  as $\cont{S}{P}$.
\end{definition}

\begin{definition}
  The \emph{extension} of a container $\cont{S}{P}$ is a functor
  $\context{\cont{S}{P}} : \Func{\Type}{\Type}$, with its action on
  objects defined as
  $\context{\cont{S}{P}}_0\ X \ddefeq (s : S) \times (P\ s \to X)$ and
  its action on morphisms as
  $\context{\cont{S}{P}}_1\ f \ddefeq \lambda (s , t) . (s , f \circ
  t)$
\end{definition}

Looking at the action on morphisms of a container, we see that it is
defined in terms of composition in $\Type$, so the functor laws
translate to the category laws of $\Type$, hence they are satisfied
strictly.

However, to define the category of sorts, we need not only
endofunctors on $\Type$ but also functors with as its domain another
category of sorts. We can generalise the notion of containers as
follows:

\begin{definition}
  A \emph{generalised container} on a category $\Cc$ is a type of
  \emph{shapes} $S : \Type$ along with a family of \emph{positions}
  $P : S \to | \Cc | $, denoted as $\cont{S}{P}$.
\end{definition}

\begin{definition}
  The \emph{extension} of a generalised container $\cont{S}{P}$ on
  $\Cc$ is a functor $\context{\cont{S}{P}} : \Func{\Cc}{\Type}$, with
  its action on objects defined as
  $\context{\cont{S}{P}}_0\ X \ddefeq (s : S) \times (\Cc(P\ s, X))$
  and its action on morphisms as
  $\context{\cont{S}{P}}_1\ f \ddefeq \lambda (s , t) . (s , f \circ
  t)$
\end{definition}

As before, the functor laws of a container depend on the category laws
of the domain of the functor. If we have a definition of a sort
category in which all the functors are given as containers, the
resulting category will satisfy the category laws strictly.

\section{Categories of algebras}

So far we have not given a precise definition of $\dialgcat{F}{G}$ for
some functors $F, G : \Cc \to \Dd$. In \cref{dialg}, we have defined
the category its objects and morphisms. The objects are defined in
terms of objects from $\Cc$ and morphisms from $\Dd$. Morphisms are
defined in terms of morphisms in $\Cc$ and $\Dd$ along with equalities
between them. If we are only interested in objects and morphisms of
$\dialgcat{F}{G}$, we need to know what the objects and morphisms of
$\Cc$ and $\Dd$ are and the actions of $F$ and $G$ on those. Once we
go beyond morphisms, we run into trouble.

To illustrate these issues, we will look at the definition of the
$\dialgcat{F}{G}$ in detail. 

\subsection{Identity morphisms}

Given an object $(X,\theta) : | \dialgcat{F}{G} |$, we define the
identity morphism as
$$
\id_{\dialgcat{F}{G}}\ (X,\theta) \ddefeq (\id_\Cc\ X , \id_0)
$$
with $\id_0$ defined as:
$$
\xymatrix{
G\ \id_\Cc \circ \theta
\ar@{-}[rr]^{\id_0}
\ar@{-}[d]_{\Gid \circ \theta}
&
&\theta \circ F\ \id_\Cc
\ar@{-}[d]^{\theta \circ \Fid}
\\
\id_\Dd \circ \theta
\ar@{-}[dr]_{\leftid_{\Dd}}
&
&\theta \circ \id_\Dd
\ar@{-}[dl]^{\rightid_{\Dd}}
\\
&\theta
&
}
$$

Unsurprisingly, the construction of identity morphisms relies on the
functor $F$ and $G$ preserving identity morphisms. Perhaps more
surprising is that the construction also relies on the identity
\emph{laws} of the category $\Dd$, \ie a category structure ``one
level up'' from identity morphisms.

In our cases, the codomain of functors $F$ and $G$ is always $\Type$
or some other category of sorts: it does not change with the number of
constructors.

\subsection{Composition}

Suppose we are given algebras
$(X,\theta),(Y,\rho)(Z,\zeta) : | \dialgcat{F}{G} |$ and morphisms
$(g,g_0) : (Y,\rho) \to (Z,\zeta)$ and
$(f,f_0) : (X,\theta) \to (Y,\rho)$. If we want to compose the two
morphisms, we need a way to glue the squares $g_0$ and $f_0$ together,
\ie we need an operation:
$$
\xymatrix{
 FX \ar[r]^\theta \ar[d]_{F f}                 \ar@{}[dr]|{f_0} &GX \ar[d]^{G f}="Gf"
&
&FY \ar[r]^\rho   \ar[d]_{F g}="Fg"            \ar@{}[dr]|{g_0} &GY \ar[d]^{G g}="Gg"
&
&FX \ar[r]^\theta \ar[d]_{F (g \circ f)}="Fgf" \ar@{}[dr]|{g_0 \circ_0 f_0} &GX \ar[d]^{G (g \circ f)}
\\ 
 FY \ar[r]_{\rho}  &GY
&
&FZ \ar[r]_{\zeta} &GZ
&
&FZ \ar[r]_{\zeta} &GZ
\ar "Gf";"Fg"
\ar "Gg";"Fgf"
}
$$
The (vertical) composition of the squares $g_0 \circ_0 f_0$ is defined as the composite:
$$
\xymatrix{
G\ (g \circ f) \circ \theta
\ar@{-}[r]^{g_0 \circ_0 f_0}
\ar@{-}[d]_{\Gcomp g f \circ \theta}
&\zeta \circ F(g \circ f)
\ar@{-}[d]^{\zeta \circ \Fcomp g f}
\\
(G\ g \circ G\ f) \circ \theta
\ar@{-}[d]_{\assoc_{\Dd}}
&\zeta \circ (F\ g \circ F\ f)
\ar@{-}[d]^{\assoc_{\Dd}}
\\
G\ g \circ (G\ f \circ \theta)
\ar@{-}[d]_{G g \circ f_0}
&(\zeta \circ F\ g) \circ F\ f
\ar@{-}[d]^{g_0 \circ F f}
\\
G\ g \circ (\rho \circ F\ f)
\ar@{-}[r]_{\assoc_{\Dd}}
&(G\ g \circ \rho) \circ F\ f
}
$$
Composition of dialgebra morphisms is then:
$$
(g,g_0) \circ (f,f_0) \ddefeq (g \circ f , g_0 \circ_0 f_0)
$$
As with identity morphisms, we notice that we need to appeal to the
functors preserving the same kind of structure we are defining here:
they need to preserve composition. We also need categorical structure
one level up from composition from the category $\Dd$, namely
associativity. 

\subsection{Identity laws}

\begin{proposition}
  \label{morphism-equality}
  Let $(f,f_0),(g,g_0)$ be two dialgebra morphisms
  $(X,\theta) \to (Y,\rho)$ in $\dialgcat{F}{G}$, then we have the
  following equivalence of equalities:
  \begin{align*}
    ((f,f_0) = (g,g_0)) &= &&(p : f = g)\\
    &\times &&(p_0 : 
              \xymatrix{
              G\ f \circ \theta
              \ar@{-}[r]^{f_0}
              \ar@{-}[d]_{G p \circ \theta}
              &\rho \circ F\ f
              \ar@{-}[d]^{\rho \circ F p}
              \\
              G\ g \circ \theta
              \ar@{-}[r]_{g_0}
              &\rho \circ F\ g
              })
  \end{align*}
\end{proposition}

\begin{proof}
  proof
\end{proof}

To give a witness for the left identity law, we need to show given:
\begin{itemize}
\item objects $(X,\theta), (Y,\rho) : |\dialgcat{F}{G}|$
\item with a morphism $(f,f_0) : (X,\theta) \to (Y,rho)$
\end{itemize}
that $\id_{\dialgcat{F}{G}}\ (Y,\rho) \circ_{\dialgcat{F}{G}} (f,f_0) = (f,f_0)$. Unfolding definitions this reduces to having to show that:
$$
(\id_{\Cc}\ Y \circ_{\Cc} f , \id_0\ \rho \circ_0 f_0) = (f,f_0)
$$
Applying \cref{morphism-equality} this is the same as giving a proof $p : \id_{\Cc}\ Y \circ f = f$ along with a square:
$$
\xymatrix{
              G\ (\id_{\Cc}\ Y \circ f) \circ \theta
              \ar@{-}[r]^{\id_0\ \rho \circ_0 f_0}
              \ar@{-}[d]_{G p \circ \theta}
              &\rho \circ F\ (\id_{\Cc}\ Y \circ f)
              \ar@{-}[d]^{\rho \circ F p}
              \\
              G\ f \circ \theta
              \ar@{-}[r]_{f_0}
              &\rho \circ F\ f
              }
$$
For $p$ we can fill in $\leftid_{\Cc}\ f$


\subsection{Associativity}

\section{Equaliser categories}

