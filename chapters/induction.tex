\chapter{Induction versus initiality}

\section{Categorical characterisation of induction}

The induction principle of an inductive type $\Tty$ gives us a way to
construct dependent functions for families defined on $\Tty$. The
family $P$ which we are eliminating into is also called the
\emph{motive} of the inductive. By providing \emph{methods} for every
constructor for this motive, the induction principle gives us a
function $(x : \Tty) \to P\ x$. Recall that in the case of the natural
numbers, the methods we have to supply have the following types:
%
\begin{itemize}
\item $m_{\natzero} : P\ \natzero$,
\item $m_{\natsucc} : (n : \natty) \to P\ n \to P\ (\natsucc\ n)$,
\end{itemize}
%
Given all this, the induction principle yields a dependent function
$s : (x : \natty) \to P\ x$ satisfying certain computation rules.

We can think of the triple $(P,m_{\natzero},m_{\natsucc})$ as a family
of algebras defined over the algebra $(\natty, \natzero, \natsucc)$:

\begin{definition}
  We define a type of algebra families for this particular category of
  algebras as:
%
  \begin{alignat*}{2}
    &\Fam_{\Alg_{\lambda X . 1 + X}} : | \Alg_{\lambda X . 1 + X} | \to \Type \\
    &\Fam_{\Alg_{\lambda X . 1 + X}} (X , \theta_0, \theta_1) &\ddefeq& (P : X \to \Type) \\
    &&\times& (m_0 : P\ \theta_0) \\
    &&\times& (m_1 : (x : X) \to P\ x \to P\ (\theta_1\ x))
  \end{alignat*}
%
\end{definition}

In order to see that this type does correspond to a notion of family,
recall that families on a type can also be represented as functions
into said type:

\begin{proposition}
Given $X : \Type$, there is an equivalence:
$$
(X \to \Type) = (Y : \Type) \times (p : Y \to X)
$$
\end{proposition}

\begin{proof}
  Let $P : X \to \Type$ be a family on $X$, we can map this to the
  pair $((x : X) \times P\ x, \pi_0)$, \ie the family's total space
  along with its projection map into its base space. In the other
  direction we map $(Y,p)$ to the preimage family
  $\lambda x . (y : Y) \times (p\ y = x)$. Checking that these two
  maps are eachother's inverses can be done by using function
  extensionality and univalence.
\end{proof}

For the aforementioned algebra families, we have a similar
equivalence:

\begin{proposition}
Given $X : \Alg_{\lambda X . 1 + X}$, there is an equivalence:
$$
\Fam_{\Alg_{\lambda X . 1 + X}}\ X = (Y : | \Alg_{\lambda X . 1 +
  X} |) \times (p : \Alg_{\lambda X . 1 + X}(Y,X))
$$
\end{proposition}

\begin{proof}
  The proof follows the same structure as the $\Type$ case. Given an
  algebra family $(P,m_0,m_1)$, we can define its ``total algebra'' as
  follows:
  $$
  \total\ (P,m_0,m_1) \ddefeq ((x : X) \times P\ x , (\theta_0, m_0), (\lambda (x,p) . (\theta_1\ x, m_1\ x\ p)))
  $$
  The projection function $\pi_0 : (x : X) \times P\ x \to X$ turns
  out to be an algebra morphism
  $\total\ (P,m_0,m_1) \to (X,\theta_0,\theta_1)$: it satisfies the
  computation rules definitionally. Let us denote this morphism as
  $\proj\ (P,m_0,m_1)$. The mapping from left to right maps
  $(P,m_0,m_1)$ to the pair
  $(\total\ (P,m_0,m_1), \proj\ (P,m_0,m_1))$.

  For the other direction we need to generalise the preimage family to
  algebras: given $(Y,\rho_0,\rho_1)$ with
  $(p,p_0,p_1) : \Alg_{\lambda X . 1 +
    X}((Y,\rho_0,\rho_1),(X,\theta_0,\theta_1))$,
  we define the following family:
  %
  \begin{align*}
  &(\ \lambda x . (y : Y) \times p\ y = x &&: X \to \Type \\
  &,\ (\rho_0 , p_0) &&: (y : Y) \times p\ y = \theta_0 \\
  &,\ \lambda x (y,z) . (\rho_1\ y , w) &&: (x : X) \to (y : Y) \times (p\ y = x) \to (y' : Y) \times (p\ y' = \theta_1\ x)\\
  &)
  \end{align*}
  % 
  where $w$ is defined as the following path:
  $$
  \xymatrix{
    p\ (\rho_1\ y) \ar@{-}[r]^-{p_1} &\theta_1\ (p\ y) \ar@{-}^-{\ap\ \theta_1\ z}[r] &\theta_1\ x
  }
  $$
\end{proof}

Given a family $P : X \to \Type$, a dependent function
$(x : X) \to P\ x$ corresponds to a \emph{section} of the projection
function $\pi_0 : (x : X) \times P\ x \to X$. As it turns out, the
corresponding notion of dependent function for an algebra family is a
dependent function along with computation rules, \ie everything we get
from the induction principle:

\begin{definition}
  Given an algebra family $(P,m_0,m_1)$, a \emph{dependent algebra
    morphism} is a dependent function $s : (x : X) \to P\ x$ equipped
  with the computation rules:
  % 
  \begin{itemize}
  \item $s_{\natzero} : s\ \natzero = m_{\natzero}$
  \item $s_{\natsucc} : (n : \natty) \to s\ (\natsucc\ n) = m_{\natsucc}\ n\ (s\ n)$
  \end{itemize}
\end{definition}

As the definitions of function into $X$ and section only refer to the
category structure, this generalises to any category. The induction
principle that gives us a dependent morphism for any family can be
therefore be phrased abstractly as follows:

\begin{definition}
  The \emph{section principle} for an object $X$ in a category $\Cc$
  says that for every $Y : | \Cc |$ and $p : \Cc(Y,X)$, there exists
  $s : \Cc(X,Y)$ and a proof of $p \circ s = \id_X$, \ie that there is
  a term of type:
  \[
    (Y : | \Cc |) \times (p : \Cc(Y,X)) \to (s : \Cc(X,Y)) \times (p \circ s = \id_X)
  \]
\end{definition}

\section{Section principle is logically equivalent to initiality}

Now we have a category theoretic characterisation of the induction
principle, we have to show that it is logically equivalent to
initiality. Assuming a bit more structure of the categories we are
working with, namely that finite limits exist, we can show that an
object satisfies the section principle if and only if it is an initial
object.

\begin{lemma}
  \label{initiality-to-section}
  Let $\Cc: \Cat$. If $X : | \Cc |$ is initial, then $X$ satisfies the
  section principle.
\end{lemma}
\begin{proof}
  Assume $X$ is initial. Given a morphism $p : Y \to X$, we need to
  produce a morphism $s : X \to Y$ such that $p \circ s = id_X$.
  Since $X$ is initial, we get a unique arrow $s : X \to Y$ such that:
  $$
  \xymatrix{
    X \ar[r]^{\id_{X}} \ar[rd]_{s} &X \\
    &Y \ar[u]_{p}
  }
  $$
  The composite has to be equal to the identity morphism on $X$, as
  that by initiality is the only endomorphism on $X$.
\end{proof}

\begin{lemma}
  \label{section-to-initiality}
  Let $\Cc: \Cat$ and assume $\Cc$ has finite limits. If $X : | \Cc |$
  satisfies the section principle, then $X$ is initial.
\end{lemma}
\begin{proof}
  Given $Y : |\Cc|$, we need to provide a unique arrow $X \to
  Y$. Consider the projection $\pi_0 : X \times Y \to X$, which is an
  arrow into $X$ and therefore has a section $s : X \to X \times
  Y$. Our candidate arrow is then the composite:
  $$
  \xymatrix{
    X \ar[r]^{s} &X \times Y \ar[r]^{\pi_1} &Y
  }
  $$
  which we have to show is unique. Using equalisers, we can show that
  any two arrows $f,g$ out of $X$ to some other object $Y$ are equal:
  \[
    \xymatrix{
      E \ar[r]^{i} &X \ar@<-.5ex>[r]_-{f} \ar@<.5ex>[r]^-{g} &Y \\
      X \ar[u]^{s} \ar[ur]_{\id_{X}}
    }
  \]
  Let $E$ be the equaliser of $f$ and $g$, then we get a projection map
  $i : E \to X$. By the section principle, this map has a section
  $s : X \to E$, hence we have:
  \begin{align*}
    f &= \id_X \circ f \\
      &= (s \circ i) \circ f \\
      &= s \circ (i \circ f) \\
      &= s \circ (i \circ g) \\
      &= (s \circ i) \circ g \\
      &= \id_X \circ g \\
      &= g
  \end{align*}
\end{proof}

Stating that an object is initial only requires us to define the type
of objects and type of morphisms of the category. This makes it an
attractive notion to internalise and work with, as defining it will
not give us any coherence issues. The section principle however, needs
a bit more structure: we need composition and identity morphisms. In
order to show that initiality and the section principle coincide, we
need even more structure: we need the identity \emph{laws} and
associativity and the existence of certain limits.

\section{Limits in categories of algebras}

In this section we will show that the categories of algebras we are
working with have products and equalisers, and hence satisfy the
assumption of~\cref{section-to-initiality}. This is done by induction
on the specification of the inductive definition, \ie by induction on
its number of constructors. We will see that we also need that the
forgetful functors into the category of sorts preserve these limits,
which we can prove simultaneously with the construction of the limits.

For the empty specification, an inductive definition with no
constructors, the resulting category of algebras is the category of
sorts. We will show that this category has the required limits:

\begin{lemma}
\label{sorts-products}
  For each sort $\Ss : \sortsty$, the category $\SortCat{\Ss}$ has
  products.
\end{lemma}

\begin{proof}
  We proceed by induction on the specification of sorts
  $\Ss : \sortsty$.  If $\Ss = \sortsnil$, then $\SortCat{\Ss} = \termcat$,
  which trivially satisfies our criteria.

  In the induction step case, we have $\SortCat{\Ss} = S_i$ for a
  category $S_i$ which is built out of the previous category of sorts
  $S_{i-1} : \Cat$ with $R_i : S_{i-1} \to \Set$. By the induction
  hypothesis $S_{i-1}$ has products and equalisers. We can then define
  products in $S_{i}$ as follows: suppose $(X,P), (Y,Q) : | S_{i} |$,
  \ie $X, Y : | S_{i-1} |$ and $P : R_i X \to \Set$,
  $Q : R_i Y \to \Set$, since $S_{i-1}$ has products, we can define:
  \[
    (X,P) \times (Y,Q) \ddefeq (X \times Y , P \times Q)
  \]
  where $P \times Q : R_i (X \times Y) \to \Set$ is defined pointwise
  as:
  \[
    (P \times Q)\ x \ddefeq P (R_i\ \pi_0\ x) \times Q (R_i\ \pi_1\ x)
  \]
  This definition satisfies the universal property of products, which
  can be shown by appealing to the universal properties of products in
  $S_{i-1}$ and $\Set$: let $(Z,T) : | S_{i} |$, then we have:
  % 
  \begin{align*}
    &&&S_i((Z,T),(X \times Y, P \times Q) \\
    &=& &\reasontext{definition of products in $S_i$} \\
    &&& (f : S_{i-1}(Z,X \times Y)) \\
    &&& \times (g : (x : R_i Z) \to T\ x \to (P \times Q)\ (R_i\ f\ x)) \\
    &=& &\reasontext{universal property of $X \times Y$ in $S_{i-1}$ and functoriality of $R_i$} \\
    &&& (f_0 : S_{i-1}(Z,X)) \times (f_1 : S_{i-1}(Z,Y)) \\
    &&& \times (g : (x : R_iZ) \to T\ x \to P\ (R_i\ f\ x) \times Q\ (R_i\ g\ x)) \\
    &=& &\reasontext{universal property of $P\ (R_i\ f\ x) \times Q\ (R_i\ g\ x)$ in $\Set$} \\
    &&& (f_0 : S_{i-1}(Z,X)) \times (f_1 : S_{i-1}(Z,Y)) \\
    &&& \times (g_0 : (x : R_iZ) \to T\ x \to P\ (R_i\ f\ x)) \\
    &&& \times (g_1 : (x : R_iZ) \to T\ x \to Q\ (R_i\ g\ x)) \\
    &=& &\reasontext{definition of products in $S_i$} \\
    &&& S_i((Z,T),(X,P)) \times S_i((Z,T),(Y,Q))
  \end{align*}
\end{proof}

Equalisers are constructed in similar way to products, however it
involves equalities between morphisms. Suppose
$(f,f'), (g,g') : S_i((X,Y),(Z,W))$, then we have by function
extensionality and using that an equality of pairs is equivalent a
pair of equalities:
% 
\begin{align*}
  ((f,f') = (g,g')) &=& &(p : f = g) \\
                    &\times& &(p' : (x : X) (y : Y\ x) \to \pathover{W}{R_i\ p\ x}{f'\ x\ y}{g'\ x\ y})
\end{align*}
% 
where we denote the action of $R_i$ on a proof of equality $p : f = g$
by $R_i(p)\ x : R_i\ f\ x = R_i(g)\ x$. Since
$f'\ x\ y : W\ (R_i\ f\ x)$ and $g'\ x\ y : W\ (R_i\ g\ x)$, we have
to transport the left hand side of the equation along the equality
$R_i\ p\ x$.

\begin{lemma}
\label{sorts-equalisers}
  For each sort $\Ss : \sortsty$, the category $\SortCat{\Ss}$ has
  equalisers.
\end{lemma}

\begin{proof}
  As before, we proceed by induction on the specification of sorts
  $\Ss : \sortsty$.  If $\Ss = \sortsnil$, then $\SortCat{\Ss} = \termcat$,
  which trivially satisfies our criteria.



  Given $(f,f'), (g,g') : S_i((X,Y),(Z,W))$, by the induction
  hypothesis we get an equaliser $E : |S_{i-1}|$ with a projection map
  $e : S_{i-1}(E,X)$. This equaliser comes equipped with a proof
  $p : f \circ e = g \circ e$. The equaliser is then defined as
  $(E,F)$ with:
  \begin{align*}
    F\ x & \ddefeq (y : Y\ (R_i\ e\ x)) \\
         & \times (\pathover{W}{R_i\ p\ x}{f'\ (R_i\ e\ x)\ y}{g'\ (R_i\ e\ x)\ y})
  \end{align*}
  with $(e,e') : S_i((E,F),(X,Y))$ the projection morphism where
  \[
    e'\ x\ (y , p) \ddefeq y
  \]
  Showing that $(f,f') \circ (e,e') = (g,g') \circ (e,e')$ is then
  straightforward. The universal property can be shown similarly to
  that of the product: we have to appeal to the universal properties
  of equalisers in $S_{i-1}$ and $\Set$.
\end{proof}

Combining this we get:

\begin{theorem}
  For each sort $\Ss : \sortsty$, the category $\SortCat{\Ss}$ has
  finite limits.
\end{theorem}

When specifying a quotient inductive-inductive type with at least one
constructor, we also need the category of algebras to have finite
products.

\begin{lemma}[Limits in the category of algebras]
  For each sort $\Ss : \sortsty$ and specification $s : \specty$, the
  category of algebras $\Alg_s$ has products and
  equalisers. Furthermore, the forgetful functor
  $U_s : \Alg_s \to \SortCat{\Ss}$ preserves finite limits.
\end{lemma}
\begin{proof}
  We prove both properties simultaneously by induction on $s : \specty$.
  For the empty quotient inductive-inductive type specification, the
  first statement is Lemma~\ref{thm:lim-sorts}, and the forgetful
  functor of the category of algebras into the category of sorts is
  the identity functor, which trivially preserves limits. By the way
  the limits here are constructed, the functor $t_i : S_i \to S_{i-1}$
  also preserves them. This means that if we have a forgetful functor
  $U : \Alg_s \to \Ss$ for some $s : \specty$, then for any
  $S_i \in \Ss$, the extension $\bar{U} : \Alg_s \to S_i$ of $U$ also
  preserves products and equalisers.

  For the induction step case, we have a specification $s : \specty$
  such that $\Alg_s : \Cat$ has products and equalisers and the
  forgetful functor $U : \Alg_s \to \Ss$ preserves them. To define a
  constructor on $s$, we need to specify its sort first. Suppose we
  have $S_i \in \Ss$, then either $S_i = 1$ or $S_i$ is built out of
  $S_{i-1}$ with some functor $R_i : S_{i-1} \to \Set$ and we have a
  forgetful functor $t_i : S_i \to S_{i-1}$. In the first case, we are
  done: adding constructors to an object in $1$ does not do
  anything. In the remainder we will assume the latter is the case.

  A 0-constructor of sort $S_i$ on a specification $s : \specty$ is
  given by a functor $F : \Alg_s \to S_i$ such that
  $t_i \circ F = t_i \circ \bar{U}$. The algebras we are interested in
  are dialgebras $(X : \Alg_s) \times (\theta : S_i(FX, \bar{U}X))$
  such that $t_i \theta = \id_{t_i (\bar{U}X)}$.

  Let $(X,\theta), (Y,\rho) : | \dialgcat{F}{\bar{U}} |$, then by the
  induction hypothesis we have that the projections map
  $\bar{U}(X \times Y) \to \bar{U}X \times \bar{U}Y$ has an inverse
  $\phi$. We can define the algebra structure on $X \times Y$ as
  follows:
  $$
  \xymatrix{ F(X \times Y) \ar[r] &
    FX \times FY \ar[ld]_{\theta \times \rho} \\
    \bar{U}X \times \bar{U}Y \ar[r]^{\phi} & \bar{U}(X \times Y) }
  $$

  If $t_i\ \theta = \id_{t_i (\bar{U}X)}$ and
  $t_i\ \rho = \id_{t_i (\bar{U}Y)}$, then the composite will also
  satisfy this property, hence we are done. By the construction of
  products, the forgetful functor out of the new category of algebras
  preserves products.

  Equalisers are constructed in an analogous fashion.

  For 1-constructors the situation is also similar: we have to appeal
  to the $\bar{U}$ preserving the limits and have to use the fact that
  an equality between dependent pairs is equivalent to a dependent
  pair of equalities.
\end{proof}

Together with \cref{initiality-to-section} and
\cref{section-to-initiality}, this immediately gives the main theorem
of this section:

\begin{theorem}
\label{thm:main}
  For each sort $\Ss : \sortsty$ and specification $s : \specty$, let $X$
  be an object in the category of algebras $\Alg_s$. Then $X$ is
  initial if and only if $X$ satisfies the section principle. \qed
\end{theorem}

In particular, this means that when implementing or formalising
quotient inductive-inductive types, one can restrict attention to the
conceptually simpler notion of initial algebra.

\section{Deriving the induction principle}
