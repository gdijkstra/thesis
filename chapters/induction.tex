\chapter{Induction versus initiality}

\section{Categorical characterisation of induction}

The induction principle of an inductive type $\Tty$ gives us a way to
construct dependent functions $(x : \Tty) \to P\ x$ for families
$P : \Tty \to \Type$. The family $P$ which we are eliminating into is
also called the \emph{motive} of the inductive.

For example, if we have a family $P : \natty \to \Type$ defined on the
natural numbers, and given
%
\begin{itemize}
\item $m_{\natzero} : P\ \natzero$,
\item $m_{\natsucc} : (n : \natty) \to P\ n \to P\ (\natsucc\ n)$,
\end{itemize}
%
the induction principle yields a dependent function
$s : (x : \natty) \to P\ x$. This dependent function comes with a
computation rule for every method:
%
\begin{itemize}
\item $s_{\natzero} : s\ \natzero = m_{\natzero}$
\item $s_{\natsucc} : (n : \natty) \to s\ (\natsucc\ n) = m_{\natsucc}\ n\ (s\ n)$
\end{itemize}
%
We can think of the triple $(P,m_{\natzero},m_{\natsucc})$ as an
\emph{algebra family} over the algebra $(\natty, \natzero, \natsucc)$,
\ie we have
$(P,m_{\natzero},m_{\natsucc}) : \Fam_{\Alg_{\lambda X . 1 + X}}
(\natty, \natzero, \natsucc)$, where
%
\begin{align*}
\Fam_{\Alg_{\lambda X . 1 + X}} (X , \theta_0, \theta_1) &\ddefeq (P : X \to \Type) \\
& \times (m_0 : P\ \theta_0) \\
&\times (m_1 : (x : X) \to P\ x \to P\ (\theta_1\ x))
\end{align*}
%
For families in $\Type$, we have the equivalence, given $X : \Type$:
$$
(X \to \Type) = (Y : \Type) \times (p : Y \to X)
$$
For the abovementioned algebra families, we have a similar
equivalence: for every $X : \Alg_{\lambda X . 1 + X}$, we have:
$$
\Fam_{\Alg_{\lambda X . 1 + X}}\ X = (X : | \Alg_{\lambda X . 1 +
  X} |) \times (p : \Alg_{\lambda X . 1 + X}(Y,X))
$$


Similarly, dependent functions $(x : X) \to P\ x$ for a predicate
$P : X \to \Type$ correspond to sections of the function $p : Y \to X$
corresponding to $P$. The notion of function into $X$ and section
readily generalises to any category. We can then state the induction
principle for an algebra as the following section principle: \emph{any
  algebra fibration has an algebra section}.

\begin{definition}
  The \emph{section principle} for an object $X$ in a category $\Cc$
  says that for every $Y : | \Cc |$ and $p : \Cc(Y,X)$, there exists
  $s : \Cc(X,Y)$ and a proof of $p \circ s = \id_X$, \ie that there is
  a term
  \[
    \sectInd : (Y : | \Cc |) \times (p : \Cc(Y,X)) \to (s : \Cc(X,Y)) \times (p \circ s = \id_X)
  \]
\end{definition}

\section{Section principle is logically equivalent to initiality}

The section principle can be stated for any category, so we do not
have to work with the details of the categories of algebras. Assuming
a bit more structure of the category, namely that finite limits, we
can show that an object satisfies the section principle if and only if
it is an initial object.

\begin{lemma}
\label{thm:initToSec}
  Let $\Cc: \Cat$. If $X : | \Cc |$ is initial, then $X$ satisfies the
  section principle.
\end{lemma}
\begin{proof}
  Assume $X$ is initial. Given a morphism $p : Y \to X$, we need to
  produce a morphism $s : X \to Y$ such that $p \circ s = id_X$.
  Since $X$ is initial, we get for any $Y$ a unique arrow $X \to
  Y$. If we postcompose this with $p$, we will get a morphism
  $X \to X$, which by initiality has to be the identity morphism.
\end{proof}

\begin{lemma}
\label{thm:secToInit}
  Let $\Cc: \Cat$ and assume $\Cc$ has finite limits. If $X : | \Cc |$
  satisfies the section principle, then $X$ is initial.
\end{lemma}
\begin{proof}
  Given $Y : |\Cc|$, we need to provide a unique arrow $X \to
  Y$. Consider the projection $\pi_1 : X \times Y \to X$, which is an
  arrow into $X$ and therefore has a section $s : X \to X \times
  Y$. Our candidate arrow is then $\pi_2 \circ s : X \to Y$, which we
  have to show is unique. Using equalisers, we can show that any two
  arrows $f,g$ out of $X$ to some other object $Y$ are equal:
\[
\xymatrix{
E \ar[r]^{i} &X \ar@<-.5ex>[r]_-{f} \ar@<.5ex>[r]^-{g} &Y \\
X \ar[u]^{s} \ar[ur]_{\id_{X}}
}
\]
Let $E$ be the equaliser of $f$ and $g$, then we get a projection map
$i : E \to X$. By the section principle, this map has a section
$s : X \to E$, hence
$f = \id_X \circ f = s \circ i \circ f = s \circ i \circ g = \id_X
\circ g = g$ holds.
\end{proof}

\section{Limits in categories of algebras}

\section{Deriving the induction principle}
