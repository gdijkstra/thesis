\chapter{Quotient inductive types}
\label{qits}

In this chapter we will give several examples of quotient inductive
types, motivating their usefulness, before we set out giving a precise
definition. We will also compare it to other notions such as quotient
types. We will discuss the status of current (partial) implementations
of quotient and higher inductive types in various proof assistants.

\section{Examples}
\label{examples}

\subsection{Interval type}
\label{int}

The simplest (non-empty) example of an inductive type with a path
constructor which is still a set, is the interval type: two point
constructors with a path constructor connecting them:
%
\begin{datatype}{\intty}{\Type}
  \constr{\intzero}{\intty} \\
  \constr{\intone}{\intty} \\
  \constr{\intseg}{\intzero = \intone}
\end{datatype}
%
It should come as no surprise that this type can be shown to be
equivalent to the unit type. However, this does not mean it is
entirely uninteresting: unlike the unit type, the interval type
implies function extensionality (\cref{int:funext}).

As the interval type can be shown to be contractible, it does not
matter whether we take the set truncation of it or not. As such, we
will work with untruncated types for the remainder of this example.

To define a function out of an inductive type, we need to say what
each constructor is mapped to. For the interval type, we first of all
need to give two points. Since every function in type theory is a
congruence, \ie we have $\ap$, these two points need to be equal to
eachother. The recursion principle is therefore as follows:
$$
\intrec : (A : \Type) (a_{\intzero} : A) (a_{\intone} : A) (a_{\intseg} : a_{\intzero} = a_{\intone})
\to \intty \to A
$$
which comes with the computation rules:
%
\begin{align*}
  \intrecbetazero : \intrec\ A\ a_{\intzero}\ a_{\intone}\ a_{\intseg}\ \intzero = a_{\intzero} \\
  \intrecbetaone : \intrec\ A\ a_{\intzero}\ a_{\intone}\ a_{\intseg}\ \intone = a_{\intone}
\end{align*}

\subsubsection{Path computation rules}
\label{int:pathcomp}

The recursion principle of an inductive type comes with a computation
rule for every constructor. So far we have only given the computation
rules for the \emph{point constructors}. The computation rule for the
path constructor is similar to those of the point constructors: it
tells us that the action of $\intrec$ on the path $\intseg$ gives us
back the $a_{\intseg}$ we put in, that is:
$$
\ap\ (\intrec\ A\ a_{\intzero}\ a_{\intone}\ a_{\intseg})\ \intseg = a_{\intseg}
$$
However, the above does not type check if $\intrecbetazero$ and
$\intrecbetaone$ are not strict equalities. We would need to transport
the left-hand side over these point computation rules. We end up with
the following square:
$$
\xymatrix{
  \intrec\ A\ a_{\intzero}\ a_{\intone}\ a_{\intseg}\ \intzero 
  \ar@{-}^{\ap\ (\intrec\ A\ a_{\intzero}\ a_{\intone}\ a_{\intseg})\ \intseg}[rr]
  \ar@{-}_{\intrecbetazero}[d]
  &&\intrec\ A\ a_{\intzero}\ a_{\intone}\ a_{\intseg}\ \intone
  \ar@{-}^{\intrecbetaone}[d]
  \\
  a_{\intzero} \ar@{-}_{a_{\intseg}}[rr] &&a_{\intone}
}
$$
A path computation rule gives us an equation between paths in type we
are eliminating into. If that type happens to be a set, then these
equations would be trivial. Hence path computation rules for
\emph{quotient} inductive types do not add anything new and will be
omitted.

\subsubsection{Induction principle}
\label{int:ind}

The induction principle gives us a way to show that for some predicate
$P$ on $\intty$, we have a dependent function $(x : \intty) \to P\ x$.
We have to give a method for each point constructor and each path
constructor. The method we need to give for the constructor $\intseg$
is not a simple path as it was with the recursion principle:
$m_{\intzero}$ and $m_{\intone}$ may have different types. We do know
that they can be related by transporting along $\intseg$, so we end up
having to give a dependent path as the method for $\intseg$:

$$
\intind : (P : \intty \to \Type) (m_{\intzero} : P\ \intzero) (m_{\intone} : P\ \intone) (m_{\intseg} : \pathover{P}{\intseg}{\intzero}{\intone}) \to (x : \intty) \to P\ x
$$

As with the recursion principle, we of course have computation rules:
\begin{align*}
  \intind\ P\ m_{\intzero}\ m_{\intone}\ m_{\intseg}\ \intzero = m_{\intzero} \\
  \intind\ P\ m_{\intzero}\ m_{\intone}\ m_{\intseg}\ \intone = m_{\intone} \\
  \apd\ (\intind\ P\ m_{\intzero}\ m_{\intone}\ m_{\intseg})\ \intseg = m_{\intseg}
\end{align*}

\subsubsection{Function extensionality}
\label{int:funext}

Even though the interval is equivalent to the unit type, it has the
surprising property that it implies function extensionality. Having an
inductive type with two inhabitants that are propositionally equal but
not definitionally allows us to represent equalities using
functions. From the recursion principle we get the following logical
equivalence for any type $A$:
$$
I \to A \logequiv (x\ y : A) \times (x = y)
$$
This is like the universal property for the interval with equivalence
weakened to logical equivalence.

\begin{proposition}
  The interval type implies function extensionality.
\end{proposition}

\begin{proof}
  Suppose we have types $A, B$ with two functions $f,g : A \to B$ and
  a family of equations $p : (x : A) \to f\ x = g\ x$. We can define
  $\tilde{p} : A \to I \to B$ as follows:
  $$
  \tilde{p}\ a \ddefeq \intrec\ B\ (f\ a)\ (g\ a)\ (p\ a)
  $$
  We can then construct the following term of type $f = g$:
  $$
  \ap\ (\lambda\ i\ a . \tilde{p}\ a\ i)\ \intseg
  $$
  This assumes that the computation rules of $\intrec$ hold
  definitionally and that the type theory satisfies the $\eta$-law for
  functions definitionally.
\end{proof}

\subsection{Quotients and colimits}

Quotient types can be realised as a higher inductive type as follows:
suppose we have a type $A : \Type$ equipped a binary relation
$R : A \to A \to \Type$, we define:
%
\begin{datatype}{\quotty{A}{R}}{\Type}
  \constr{\quotproj{\_}}{A \to \quotty{A}{R}} \\
  \constr{\quoteq}{(x\ y : A) \to R\ x\ y \to \quotproj{x} = \quotproj{y}}
\end{datatype}
%
The elimination principle of the quotient $\quotty{A}{R}$ is as
follows:
%
\begin{align*}
&\quotind{A}{R} : \\
&(P : \quotty{A}{R} \to \Type) \\
&(m_{\quotproj{\_}} : (a : A) \to P\ \quotproj{a}) \\
&(m_{\quoteq} : (x\ y : A) (p : R\ x\ y) \to \pathover{P}{\quoteq\ x\ y\ p}{m_{\quotproj{\_}}\ x}{m_{\quotproj{\_}}\ y})\\
&\to (x : \quotty{A}{R}) \to P\ x
\end{align*}
%
Unlike with the interval types, whether we truncate or not does make a
difference here. Ordinarily, one would expect $A$ to be a set and $R$
to be $\Prop$-valued. These restrictions are not enough to ensure the
quotient will also be a set. If we take $A$ to be the unit type and
quotient it by the relation that is constantly the unit type as well,
$\quotty{A}{R}$ will be equivalent to the circle.

It is not necessary to require $R$ to be an equivalence
relation. Since the path constructor $\quoteq$ constructs elements of
an identity type, we can apply symmetry and other operations to the
paths constructed by $\quoteq$. We effectively take the
reflexive-symmetric-transitive closure of $R$.

In the category of sets, one can construct colimits by using
quotients. The same is true for types: we can construct coequalisers
using quotient types. In the untruncated setting, these coequalisers
will be \emph{homotopy} coequalisers. Since we already have arbitrary
coproducts ($\Sigma$-types), we can then go on and construct arbitrary
colimits using these two building blocks.

\subsubsection{Quotient types implement coequalisers}
If we have quotient types, we use them to define colimits such as
coequalisers: suppose $A, B : \Type$ with $f, g : A \to B$ two
functions, we can define a relation $R$ on $B$:
$$
R\ x\ y \ddefeq (z : A) \times (x = f\ z) \times (y = g\ z)
$$

\begin{proposition}
  The quotient $\quotty{B}{R}$ is the coequaliser of $f$ and $g$, \ie
  $\quotproj{\_} : B \to \quotty{B}{R}$ satisfies
  $\quotproj{\_} \circ f = \quotproj{\_} \circ g$ and $\quotty{B}{R}$
  satisfies the appropriate universal property.
\end{proposition}

\begin{proof}
  We can show that the type of $\quoteq$ is equivalent to
  $\quotproj{\_} \circ f = \quotproj{\_} \circ g$ by the following
  equational reasoning:
  %
  \begin{align*}
    &&&(x\ y : B) \to (z : A) \times (f\ z = x) \times (y = g\ z) \to \quotproj{x} = \quotproj{y} \\
    &=& &\reasontext{currying and singleton contraction} \\
    &&& (z : A) \to \quotproj{f\ z} = \quotproj{g\ z} \\
    &=& &\reasontext{function extensionality} \\
    &&& \quotproj{\_} \circ f = \quotproj{\_} \circ g
  \end{align*}
  % 
  The universal property can be shown to follow directly from the
  elimination principle.
\end{proof}
\subsubsection{Coequalisers implement quotient types}

We can also define coequalisers directly as a higher inductive
type. Given types $A, B$ with functions $f, g : A \to B$, we define:
%
\begin{datatype}{\coeqty{f}{g}}{\Type}
  \constr{\coeqproj}{B \to \coeqty{f}{g}} \\
  \constr{\coeqeq}{(x : A) \to \coeqproj\ (f\ x) = \coeqproj\ (g\ x)}
\end{datatype}
%
which comes with the following elimination principle:
%
\begin{align*}
  &\coeqind{f}{g} : \\
  &(P : \coeqind{f}{g} \to \Type) \\
  &(m_{\coeqproj} : (b : B) \to P\ (\coeqproj\ b)) \\
  &(m_{\coeqeq} : (a : A) \to \pathover{P}{\coeqeq\ a}{m_{\coeqproj}\ (f\ a)}{m_{\coeqproj}\ (g\ a)}) \\
  &(x : \coeqind{f}{g}) \to P\ x
\end{align*}
%
Given a type $A : \Type$ and a relation $R : A \to A \to \Type$, we
can define a type $\tilde{R} \ddefeq (x\ y : A) \times R\ x \ y$ which
has two projections $\pi_0, \pi_1 : \tilde{R} \to A$. We can then take
the coequaliser of these two functions:
$$
\xymatrix{
\tilde{R} \ar@<-.5ex>[r]_{\pi_0} \ar@<.5ex>[r]^{\pi_1} &A \ar[r]^{\coeqproj} &\coeqty{\pi_0}{\pi_1}
}
$$

\begin{proposition}
  The coequaliser $\coeqty{\pi_0}{\pi_1}$ is the quotient of $A$ by
  $R$.
\end{proposition}

\begin{proof}
  We have to show that $\coeqty{\pi_0}{\pi_1}$ satisfies the
  elimination principle of the quotient $\quotty{A}{R}$.
\end{proof}

We have shown that coequalisers and quotient types can be phrased in
terms of eachother. We can go a bit further and state that they are
equivalent when seen as families.

\subsection{Infinitely branching trees}

The examples we have seen so far seem to be expressible quotients of
ordinary inductive types. Not all quotient inductive types have this
property. One example where we can observe a difference is with the
type of infinitely branching trees modulo permutation. We define the
following inductive type:
%
\begin{datatype}{\treety}{\Set}
  \constr{\treeleaf}{\treety} \\
  \constr{\treenode}{(\natty \to \treety) \to \treety} \\
  \constr{\treeperm}{(f : \natty \to \treety) (\phi : \natty \to \natty) \to \isequiv\ \phi \to \treenode\ f = \treenode\ (f \circ \phi)}
\end{datatype}

The $\treeperm$ constructor tells that two nodes
$f, f' : \natty \to \treety$ are considered to be equal if there
exists a permutation $\phi$ such that $f = f' \circ \phi$.

We can try and define this type as a quotient of the type without path
constructor $\treeperm$:
%
\begin{datatype}{\pretreety}{\Set}
  \constr{\pretreeleaf}{\pretreety} \\
  \constr{\pretreenode}{(\natty \to \pretreety) \to \pretreety}
\end{datatype}
%
with the relation defined inductively as:
%
\begin{datatype}{\treerelty{\_}{\_}}{\pretreety \to \pretreety \to \Set}
  \constr{\treerelnode}{(f : \natty \to \pretreety) (\phi : \natty \to \natty) \to \isequiv\ \phi \to \treerelty{\pretreenode\ f}{\pretreenode\ (f \circ \phi)}}  
\end{datatype}
%
If we then look at the ``constructors'' of the quotient $\quotty{\pretreety}{\treerel}$, we have the following:
%
\begin{align*}
  \quotproj{\pretreeleaf} : \quotty{\pretreety}{\treerel} \\
  \quotproj{\_} \circ \pretreenode : (\natty \to \pretreety) \to \quotty{\pretreety}{\treerel}
\end{align*}
%
The latter does not have the same type as $\treenode$. With the
quotient inductive definition, the induction is performed
``simultaneously'' with the quotienting. This is also where $\treety$
differs from our previous examples in which there were no recursive
occurrences in the point constructors.

Lifting the $\treenode$ constructor to the quotient is problematic. It
seems we need an inverse to the projection function
$\quotproj{\_} : \pretreety \to \quotty{\pretreety}{\treerel}$, which
we cannot expect to exist.

Note however that if we had \emph{finitely} branching trees, we need
not have such an inverse. In the finite case, we can apply the
recursion principle to each argument and then apply the $\pretreenode$
and $\quotproj{\_}$ constructors.

In the infinitely branching case we can define the lifting of
$\treenode$ if we have the axiom of choice at our disposal:

\begin{proposition}
  Assuming the axiom of choice, we can define a function
  $(\natty \to \quotty{\pretreety}{\treerel}) \to
  \quotty{\pretreety}{\treerel}$.
\end{proposition}

\begin{proof}
  We can show that the function $\quotproj{\_}$ is a surjection, \ie
  using induction on $\quotty{\pretreety}{\treerel}$ we can define a
  dependent function:
  $$
  (y : \quotty{\pretreety}{\treerel} \to \proptrunc{(x : \pretreety) \times \quotproj{x} = y}
  $$
  The axiom of choice gives us an inverse to $\quotproj{\_}$:
  $$
  s : \quotty{\pretreety}{\treerel} \to \pretreety
  $$
  satisfying for any $x : \quotty{\pretreety}{\treerel}$,
  $\quotproj{s\ x} = x$. We can then go on and lift the constructor
  $\pretreenode$ to the quotient:
  $$
  \widetilde{\pretreenode} f \ddefeq \quotproj{\pretreenode\ (s \circ f)}
  $$
\end{proof}

In the presence of choice, it seems that
$\quotty{\pretreety}{\treerel}$ and $\treety$ are equivalent. The
axiom of choice in type theory is a constructive taboo in and of
itself, it also has the law of excluded middle as a consequence
\cite{Diaconescu1975}. Going in the other direction, having an inverse
to the quotient projection function in general implies the axiom of
choice \cite{Hofmann1995}. Having quotient inductive types available
allows us to work with definitions such as $\treety$, avoiding the
need for choice principles to use them.

\subsection{Cauchy reals}

Another situation where we benefit from the ability to define an
inductive type and ``at the same time'' quotient it is when defining
the real numbers. If we want to define them as equivalence classes of
Cauchy sequences of rational numbers, we traditionally run into
problems. To show that this definition yields a version of the reals
that is Cauchy complete, we need to have countable choice at our
disposal.

Instead of taking that approach, one can define the reals as the
following quotient inductive-inductive definition:

\begin{sorts}
  \sortnamety{\realty}{\Set} \\
  \sortnamety{\realrel{\_}{\_}}{\realty \to \realty \to \Set}
\end{sorts}

\begin{datatype}{\realty}{}
  \constrdots
\end{datatype}

\begin{datatype}{\realrel{\_}{\_}}{}
  \constrdots
\end{datatype}

% \begin{align*}
% &\data\ \Real\ \where \\
% &\ \ \ \ \rat : \Rat \to \Real \\
% &\ \ \ \ \lim : (\AB{x} : \RatP \to \Real) \to ((\AB{\delta}\ \AB{\epsilon} : \RatP) \to \AB{x}\ \AB{\delta} \sim_{\AB{\delta} + \AB{\epsilon}} \ \AB{x}\ \AB{\epsilon}) \to \Real \\
% &\ \ \ \ \eq : (\AB{u}\ \AB{v} : \Real) \to ((\AB{\epsilon} : \RatP) \to \AB{u} \sim_{\AB{\epsilon}} \AB{v}) \to \AB{u} = \AB{v} \\
% \\
% &\data\ \AgdaDatatype{\_\sim_{\_}\_}\ \where \\
% &\ \ \ \ \ \ \vdots \\
% &\ \ \ \ \limrat : (\AB{x} : \RatP \to \Real) (p : (\AB{\delta}\ \AB{\epsilon} : \RatP) \to \AB{x}\ \AB{\delta} \sim_{\AB{\delta} + \AB{\epsilon}} \ \AB{x}\ \AB{\epsilon}) \\
% &\ \ \ \ \ \ \ \ \ \ \ \ (\AB{r} : \Rat) (\AB{\epsilon}\ \AB{\delta} : \RatP) \\
% &\ \ \ \ \ \ \ \to \AB{x}\ \AB{\delta} \sim_{\AB{\epsilon} - \AB{\delta}} \rat\ \AB{r} \to (\lim\ \AB{x}\ \AB{p}) \sim_{\AB{\epsilon}} (\rat\ \AB{r}) \\
% &\ \ \ \ \ \ \vdots
% \end{align*}
% %

\subsection{Syntax of type theory}

\begin{sorts}
  \sortnamety{\ttconty}{\Set} \\
  \sortnamety{\tttyty}{\ttconty \to \Set}
\end{sorts}

\begin{datatype}{\ttconty}{}
  \constr{\ttnil}{\ttconty} \\
  \constr{\ttcons{\_}{\_}}{(\Gamma : \ttconty) \to \tttyty\ \Gamma \to \ttconty}
\end{datatype}


\begin{datatype}{\tttyty}{}
  \constr{\ttzero}{(\Gamma : \ttconty) \to \tttyty\ \Gamma} \\
  \constr{\ttone}{(\Gamma : \ttconty) \to \tttyty\ \Gamma} \\
  \constr{\ttpi}{(\Gamma : \ttconty) (A : \tttyty\ \Gamma) \to \tttyty\ (\ttcons{\Gamma}{A}) \to \tttyty\ \Gamma}
\end{datatype}

\begin{sorts}
  \sortnamety{\tttmty}{(\Gamma : \ttconty) \to \tttyty\ \Gamma \to \Set}
\end{sorts}

\begin{datatype}{\tttmty}{}
  \constr{\ttapp}{(\Gamma : \ttconty) (A : \tttyty\ \Gamma) (B : \tttyty\ (\ttcons{\Gamma}{A})) \to %
    \tttmty\ \Gamma\ (\ttpi\ A\ B) \to \tttmty\ (\ttcons{\Gamma}{A})\ B} \\
  \constr{\ttlam}{(\Gamma : \ttconty) (A : \tttyty\ \Gamma) (B : \tttyty\ (\ttcons{\Gamma}{A})) \to %
    \tttmty\ (\ttcons{\Gamma}{A}) B \to \tttmty\ \Gamma\ (\ttpi\ A\ B)} \\
\end{datatype}

% %\intertext{Equations on terms:}
% %  &\ \ \ \ \ \ \vdots \\
%   &\ \ \ \ \AgdaInductiveConstructor{\Pi\beta} : (\AB{\Gamma} : \Con) (\AB{A} : \Ty\ \AB{\Gamma}) (\AB{B} : \Ty\ (\AB{\Gamma}\AgdaInductiveConstructor{,}\, \AB{A})) \\
%   &\ \ \ \ \ \     \to (\AB{t} : \Tm\ (\AB{\Gamma}, \AB{A})\ \AB{B}) \to \app\ \AB{\Gamma}\ \AB{A}\ \AB{B}\ (\lam\ \AB{\Gamma}\ \AB{A}\ \AB{B}\ \AB{t}) = \AB{t} \\
%   &\ \ \ \ \ \ \vdots
% \end{align*}
% %

\section{Implementation}

One way to ``implement'' a particular quotient inductive type, is by
postulating its constructors along with the eliminator with its
computation rules. The downside of such an approach is that as soon as
we want to use it and not just define it, we will have to manually
``call'' the computation rules.

A first approximation to implement quotient inductive types (or any
higher inductive type) in a more practical manner was discovered
by~\cite{Licata2011}. The idea is to leverage the inductive type
mechanism of Agda itself and add path constructors as postulates. One
can then define the eliminator by pattern matching on the inductive
type, adding the computation rule for the path constructors as
postulates. Care has to be taken to not expose the constructors of the
type in such a way that one can circumvent the eliminator and use
pattern matching directly, in a possible unsound way.

Having definitional computation rules for the point constructors is a
big improvement over everything being propositional. However, the path
computation rules still only hold propositionally.

In order to experiment with higher inductive types and cubical type
theory in Agda, rewrite rules have been introduced. We can denote any
relation as being a rewrite relation, for example if we want to
rewrite according to $\_=\_$, we write:
$$
\AgdaPragma{BUILTIN\ REWRITE\ \_=\_}
$$
Suppose we have a term $p : x = y$ that we want to add as a
definitional equality, \ie we want Agda to always rewrite $x$ to $y$,
then we write:
$$
\AgdaPragma{REWRITE\ p}
$$
Using these, a type such as the interval can be
``implemented'' as follows:
\begin{align*}
  &\postulate \\
  &\ \ \ \ \intzero : \intty \\
  &\ \ \ \ \intone : \intty \\
  &\ \ \ \ \intseg : \intzero = \intone \\
  &\ \ \ \ \intrec : (A : \Type) (a_{\intzero} : A) (a_{\intone} : A) (a_{\intseg} : a_{\intzero} = a_{\intone}) \\
  &\ \ \ \ \ \ \ \ \to \intty \to A \\
  &\ \ \ \ \intrecbetazero : (A : \Type) (a_{\intzero} : A) (a_{\intone} : A) (a_{\intseg} : a_{\intzero} = a_{\intone}) \\
  &\ \ \ \ \ \ \ \ \to \intrec\ A\ a_{\intzero}\ a_{\intone}\ a_{\intseg}\ \intzero = a_{\intzero} \\
  &\ \ \ \ \intrecbetaone : (A : \Type) (a_{\intzero} : A) (a_{\intone} : A) (a_{\intseg} : a_{\intzero} = a_{\intone})  \\
  &\ \ \ \ \ \ \ \ \to \intrec\ A\ a_{\intzero}\ a_{\intone}\ a_{\intseg}\ \intone = a_{\intone}\\
  \\
  &\AgdaPragma{REWRITE\ \intrecbetazero} \\
  &\AgdaPragma{REWRITE\ \intrecbetaone}
\end{align*}
Obviously these rewrite pragmas are not safe in anyway: we can easily
add rules that make the type checker loop or that allow us to inhabit
the empty type.

\subsection{Cubical type theory}

\section{Related concepts}

As mentioned, quotient inductive-inductive definitions are related to
quotient types. Quotient types have been studied in several forms in
type theory.

Inductive definitions with equalities have been studied in Miranda
(data types with laws) and as non-free data types in 

One way to think about quotient inductive-inductive definitions is to
view the constructors as atoms and function symbols of some (algebraic)
theory. The path constructors are then equations on the theory. The
types then give us the syntactic model of that particular theory. One
difference with quotient inductive-inductive definitions is that we
can mix point and path constructors, as opposed to first listing all
the point constructors and then listing all the path
constructors. This means that point or path constructors may also
refer to a previous path constructor. For quotient inductive-inductive
definitions being able to do this may turn out not to be terribly
interesting, but when we move to higher inductive types, being able to
do so is essential: without it we would not be able to give the
definition of things like the torus in the usual way.

Since quotient inductive-inductive definitions can be thought of as
syntactic models of certain theories, concepts such as monads,
essentially algebraic theories and Lawvere theories are related.